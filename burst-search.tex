\subsection{Burst Search}
\label{sec:burstsearch}

%\subsubsection{Burst Search in FRETBursts}
%\label{sec:burstsearch_code}

Following background estimation, burst search is the next step of
the analysis.
In FRETBursts, a standard burst search using a single photon stream
(see section~\ref{sec:burstsearch_intro}) is performed by calling the
\verb|Data.burst_search| method
(\href{http://fretbursts.readthedocs.org/en/latest/data_class.html#fretbursts.burstlib.Data.burst_search}{link}).
For example, the following command:

\begin{lstlisting}
d.burst_search(F=6, m=10, ph_sel=Ph_sel('all'))
\end{lstlisting}

performs a burst search on all photons
(\verb|ph_sel=Ph_sel('all')|), with a count rate threshold equal to 6 times the
local background rate (\verb|F=6|), using 10 consecutive photons to compute the
local count rate (\verb|m=10|).
A different photon stream, threshold ($F$) or number of photons $m$ can be selected 
by passing different values. 
These parameters are good general-purpose starting point for smFRET analysis 
but can they can be adjusted if needed.

Note that the previous burst search does not perform any burst size selection 
(however, by definition, the minimum bursts size is effectively $m$).
An additional parameter $L$ can be passed to impose a minimum burst
size before any correction.
However, it is recommended to select bursts only after background corrections
are applied, as discussed in the next section~\ref{sec:burstsel}.

It might sometimes be useful to specify a fixed photon-rate threshold, instead
of a threshold depending on the background rate, as in the previous example. In
this case, instead of $F$, the argument \verb|min_rate_cps| can be used to
specify the threshold (in counts-per-second). For example, a burst search with 
a 50~kcps threshold is performed as follows:

\begin{lstlisting}
d.burst_search(min_rate_cps=50e3, m=10, 
               ph_sel=Ph_sel('all'))
\end{lstlisting}

Finally, to perform a DCBS burst search (or in general an AND gate burst search,
see section~\ref{sec:burstsearch_intro}) we use the function
\verb|burst_search_and_gate|
(\href{http://fretbursts.readthedocs.org/en/latest/plugins.html#fretbursts.burstlib_ext.burst_search_and_gate}{link}), 
as illustrated in the following example:

\begin{lstlisting}
d_dcbs = bext.burst_search_and_gate(d, F=6, m=10)
\end{lstlisting}

The last command puts the burst search results in a new copy of the \verb|Data| variable \verb|d|
(in this example, the copy is called \verb|d_dcbs|).
Since FRETBursts shares the timestamps and detectors arrays between
different copies of \verb|Data| objects, the memory usage is minimized, even when 
several copies are created. 

\paragraph{Python details}
Note that, while \verb|.burst_search()| is a method of \verb|Data|,
\verb|burst_search_and_gate| is a function in the \verb|bext| module
taking a \verb|Data| object as a first argument and returning a new
\verb|Data| object.

The function \verb|burst_search_and_gate| accepts optional arguments,
\verb|ph_sel1| and \verb|ph_sel2|, whose default values correspond to the
classical DCBS photon stream selection (see section~\ref{sec:burstsearch_intro}).
These arguments can be specified to select different photon streams than those used in
a classical DCBS.

The \verb|bext| module (\href{http://fretbursts.readthedocs.org/en/latest/plugins.html}{link}) 
collects ``plugin'' functions that provides additional algorithms 
for processing \verb|Data| objects. 

\subsection{Bursts Corrections}
\label{sec:corrcoeff}

In \usalex, there are 3 important correction parameters: $\gamma$-factor, donor leakage into the acceptor channel
and acceptor direct excitation by the donor excitation laser~\cite{Lee_2005}.
These corrections can be applied to burst data by simply assigning values to the respective 
\verb|Data| attributes:

\begin{lstlisting}
d.gamma = 0.85
d.leakage = 0.15
d.dir_ex = 0.08
\end{lstlisting}

These attributes can be assigned either before or after the burst search. In the
latter case, existing burst data is automatically updated using the new
correction parameters. 

These correction factors can be used to display corrected FRET distributions.
However, when the goal is to fit the FRET efficiency of sub-populations, 
it is simpler to fit the uncorrected FRET histogram (i.e. background-corrected
proximity ratio) and then correct the fitted FRET efficiency (see SI in~\cite{Lee_2005}, SI).
Correcting the PR after fitting (instead of correcting the data in each burst) 
avoids distortion of the FRET distribution and keeps peaks of
static FRET subpopulations closer to the ideal Binomial statistics~\cite{Gopich_2007}.

FRETBursts implements the correction formulas for $E$ and $S$ in the functions
\verb|fretmath.correct_E_gamma_leak_dir| and \verb|fretmath.correct_S|
(\href{http://fretbursts.readthedocs.org/en/latest/fretmath.html}{link}). 
A derivation of these correction formulas (using computer-assisted algebra) 
can be found online as an interactive notebook (\href{http://nbviewer.jupyter.org/github/tritemio/notebooks/blob/master/Derivation%20of%20FRET%20and%20S%20correction%20formulas.ipynb}{link}).
