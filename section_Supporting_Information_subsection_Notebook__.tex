\section{Supporting Information}

\subsection{Notebook Workflow}
FRETBursts has been developed with the goal of facilitating computational reproducibility
of the performed data analysis~\cite{Buckheit_1995}. For this reason,
the preferential way of using FRETBursts is by executing one of the tutorials
which are in the form of Jupyter notebooks~\cite{Shen_2014}.
Jupyter (formerly IPython) notebooks are web-based documents which contain both
code and rich text (including equations, hyperlinks, figures, etc...).
FRETBursts tutorials are notebooks which can be re-executed,
modified or used to process new data files with minimal modifications.
The ``notebook workflow''~\cite{Shen_2014} not only facilitates
the description of the analysis (by integrating the code in a rich document)
but also greatly enhance its reproducibility by storing an execution trail
that includes software versions, input files, parameters, commands and all
the analysis results (text, figures, tables, etc...).

The Jupyter Notebook environment streamlines FRETBursts execution (compared to
a traditional script and terminal based approach) and allows
FRETBursts to be used even without prior python knowledge.
The user only needs to get familiar with the
notebook graphical environment, in order to be able to navigate and run the notebooks.
The list of FRETBursts notebooks can be found in the
\verb|FRETBursts_notebooks| repository on GitHub
(\href{https://github.com/tritemio/FRETBursts_notebooks}{link}).


\subsection{Development and Contributions}
\label{sec:dev}
Errors are an inevitable reality in any reasonably complex software. It is 
therefore critical to implement countermeasures to
minimize the probability of introducing bugs and their potential impact~\cite{Prli__2012, Wilson_2014}.
We strive to follow modern best-practices in software development which are summarized 
below.

FRETBursts (and the entire python ecosystem it depends on) is open source
and the source code is fully available for any scientist to study,
review and modify.
The open source nature of FRETBursts and of the python ecosystem,
not only makes it a more transparent, reviewable platform
for scientific data analysis, but also allows
to leverage state-of-the-art online services as GitHub (\href{http://https://github.com}{link}) for hosting,
issues tracking and code reviews, TravisCI
(\href{https://travis-ci.org}{link}) for continuous integration
(i.e. automated test suite execution on multiple platforms after each commit)
and \href{https://readthedocs.org/}{ReadTheDocs.org} for automatic documentation building and hosting.
All these services would be extremely costly, if available \textit{tout court},
for a proprietary software or platform~\cite{Freeman_2015}.

We highly value source code readability, a property which can 
reduce the number of bugs by facilitating understanding and verifying the code.
For this purpose, FRETBursts code-base is well commented (more that 35\%
of source code), 
follows the PEP8 python code style rules (\href{https://www.python.org/dev/peps/pep-0008/}{link}),
and has docstrings in napoleon format (\href{http://sphinxcontrib-napoleon.readthedocs.org/}{link}).

Reference documentation is built with Sphinx (\href{http://sphinx-doc.org/}{sphinx-doc.org})
and all the API documents are automatically generated from docstrings.
On each commit, documentation is automatically built and deployed on
\href{https://readthedocs.org/}{ReadTheDocs.org}.

Unit tests cover most of the core algorithms, ensuring consistency and
minimizing the probability of introducing bugs. 
The TravisCI (\href{http://travis-ci.org}{link}) continuous integration service, 
executes the full test suite on each commit, timely reporting errors.
As a rule, whenever a bug is discovered, the  fix also includes a new test
to ensure that the same bug cannot happen in the future.
In addition to the unit tests, we include a regression-test notebook
(\href{https://github.com/tritemio/FRETBursts/blob/master/notebooks/dev/tests/FRETBursts%20-%20Regression%20tests.ipynb}{link})
to easily compares numerical results between two versions of FRETBursts. 
Additionally, the tutorials themselves are executed before each release as
an additional test layer to ensure that no errors or regressions are introduced.

FRETBursts is openly developed using the GitHub platform.
The authors encourage users to use GitHub issues for questions, discussions
and bug reports, and to submit patches through GitHub pull requests.
Contributors of any level of expertize are welcome in the projects
and publicly acknowledged.
Contributions can be as simple as pointing out deficiencies in the 
documentation but can also be bug reports or corrections to 
the documentation or code. Users willing to implement
new features are encouraged to open an Issue on GitHub and to submit
a Pull Request. The open source nature of FRETBursts guarantees that
contributions will remain available to the entire single-molecule 
community.
