\subsection{Background Estimation}
\label{sec:bg_calc}

The first step of smFRET analysis involves estimating background rates.
For example, to compute the background every 30~s, using a minimal inter-photon
delay threshold of 2~ms for the photon stream comprising all photons, the corresponding command is:

\begin{lstlisting}
d.calc_bg(bg.exp_fit, time_s=30, tail_min_us=2000)
\end{lstlisting}

The first argument (\verb|bg.exp_fit|) is the function used to fit the
background rate for each photon stream (see section~\ref{sec:bg_intro}).
The function
\verb|bg.exp_fit| estimates the background using a maximum likelihood estimation
(MLE) of the delays distribution.
The second argument, \verb|time_s|, is the duration of the
\textit{background period} (section~\ref{sec:bg_intro}) and the third, \verb|tail_min_us|,
is the minimum inter-photon delay to use when fitting the distribution to the specified model function.
To use different thresholds for each photon stream we pass a
tuple (i.e. a comma-separated list of values, \href{https://docs.python.org/3.5/tutorial/datastructures.html#tuples-and-sequences}{link}) instead of a scalar.
The recommended approach is however automating the choice of threshold using
\verb|tail_min_us='auto'| using an heuristic algorithm which is described in 
\textit{Background estimation} section of the μs-ALEX tutorial 
(\href{http://nbviewer.jupyter.org/github/tritemio/FRETBursts_notebooks/blob/master/notebooks/FRETBursts%20-%20us-ALEX%20smFRET%20burst%20analysis.ipynb#Background-estimation}{link}).
Finally, it is possible to use a slower but rigorous approach to find the optimal
threshold as described in SI~\ref{sec:bg_opt_th}.

FRETBursts provides two kinds of plots to represent the background. One is the histograms
of inter-photon delays compared to the fitted exponential distribution, shown in 
figure~\ref{fig:bg_dist_all}) (see section~\ref{sec:bg_intro} for details on the inter-photon distribution). 
This plot is created with the command:

\begin{lstlisting}
dplot(d, hist_bg, period=0)
\end{lstlisting}

The argument \verb|period| is an integer specifying the background period to be plotted.
When omitted, the default is 0, i.e. the first period is chosen.
Figure~\ref{fig:bg_dist_all} allows to quickly identify pathological cases where the 
background fitting procedure returns unreasonable values. 

The second background-related plot represents a timetrace of background rates, 
as shown in figure~\ref{fig:bg_timetrace}. This plot allows monitoring background rate variations
occurring during the measurement and is obtained with the command:

\begin{lstlisting}
dplot(d, timetrace_bg)
\end{lstlisting}

Normally, samples should have a fairly constant background rate as a function of time
as in figure~\ref{fig:bg_timetrace}(a). However, sometimes, non-ideal
experimental conditions can yield a time-varying background rate, as illustrated in
figure~\ref{fig:bg_timetrace}(b).
A possible reason for the observed behavior could be buffer evaporation from an open sample 
(or poorly sealed) observation chamber. Alternatively,
cover-glass impurities can contribute to the background.
These impurities tend to bleach on timescales of minutes resulting in
background variations during the course of the measurement.

\paragraph{Python details} 

The estimated background rates are stored in the \verb|Data| attributes
\verb|bg_dd|, \verb|bg_ad| and \verb|bg_aa|, corresponding to photon
streams \verb|Ph_sel(Dex='Dem')|, \verb|Ph_sel(Dex='Aem')| and \verb|Ph_sel(Aex='Aem')|
respectively.
These attributes are lists of arrays (one array per excitation spot).
The arrays contain the estimated background rates in the different time trace windows.
Additional background fitting functions (e.g. least-square fitting of inter-photon delay
histogram) are available in \verb|bg| namespace 
(i.e. the \verb|background| module, \href{http://fretbursts.readthedocs.org/en/latest/background.html}
{link}). 
