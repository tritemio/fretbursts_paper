\subsection{Background Estimation}
\label{sec:bg_calc}

The first step of smFRET analysis involves estimating background rates.
For example, the following command:

% Don't split command on two lines for PLOS
\begin{lstlisting}
d.calc_bg(bg.exp_fit, time_s=30, 
          tail_min_us='auto')
\end{lstlisting}

\noindent estimates the background rates in windows of 30~s
using the default iterative algorithm for choosing the
fitting threshold (section~\ref{sec:bg_intro}).  % PLOS: remove section and use nameref
Beginner users can simply use the previous command and 
proceed to burst search (section~\ref{sec:burstsearch}). % PLOS: remove section and use nameref
For more advanced users, this section provides details on 
the different background estimation and plotting functions 
provided by FRETBursts.

First, we show how to estimate the background every 30~s, 
using a fixed inter-photon delay threshold of 2~ms 
(the same for all photon streams):

\begin{lstlisting}
d.calc_bg(bg.exp_fit, time_s=30, tail_min_us=2000)
\end{lstlisting}

The first argument (\verb|bg.exp_fit|) is the function used to fit the
background rate for each photon stream (see section~\ref{sec:bg_intro}).
The function
\verb|bg.exp_fit| estimates the background using a maximum likelihood estimation
(MLE) of the delays distribution.
The second argument, \verb|time_s|, is the duration of the
\textit{background period} (section~\ref{sec:bg_intro}) and the third, \verb|tail_min_us|,
is the minimum inter-photon delay to use when fitting the distribution to the specified model function.
To use different thresholds for each photon stream we pass a
tuple (i.e. a comma-separated list of values, \href{https://docs.python.org/3.5/tutorial/datastructures.html#tuples-and-sequences}{link}) instead of a scalar.
However, the recommended approach is to choose the threshold automatically using
\verb|tail_min_us='auto'|. This approach uses an heuristic algorithm described in the
\textit{Background estimation} section of the μs-ALEX tutorial 
(\href{http://nbviewer.jupyter.org/github/tritemio/FRETBursts_notebooks/blob/master/notebooks/FRETBursts%20-%20us-ALEX%20smFRET%20burst%20analysis.ipynb#Background-estimation}{link}).
Finally, it is possible to use a rigorous but slower approach to find an optimal
threshold as described in SI~\ref{sec:bg_opt_th}.

FRETBursts provides two kinds of plots to represent the background. One shows the histograms
of inter-photon delays compared to the fitted exponential distribution, shown in
figure~\ref{fig:bg_dist_all}) (see section~\ref{sec:bg_intro} for details on the inter-photon distribution).
This plot is created with the command:

\begin{lstlisting}
dplot(d, hist_bg, period=0)
\end{lstlisting}

This command reflects the general form of plotting commands in FRETBursts
as described in SI~\ref{sec:plotting}.
Here we only note that the argument \verb|period| is an integer specifying the background
period to be plotted (when omitted, the default is 0, i.e. the first period).
Figure~\ref{fig:bg_dist_all} allows to quickly identify pathological cases where the
background fitting procedure returns unreasonable values.

The second background-related plot represents a timetrace of background rates,
as shown in figure~\ref{fig:bg_timetrace}. This plot allows monitoring background rate variations
occurring during the measurement and is obtained with the command:

\begin{lstlisting}
dplot(d, timetrace_bg)
\end{lstlisting}

Normally, samples should have a fairly constant background rate as a function of time
as in figure~\ref{fig:bg_timetrace}(a). However, sometimes, non-ideal
experimental conditions can yield a time-varying background rate, as illustrated in
figure~\ref{fig:bg_timetrace}(b).
A possible reason for the observed behavior could be buffer evaporation from an open sample
(we strongly recommend using a sealed 
observation chamber whenever possible). Additionally,
cover-glass impurities can contribute to the background.
These impurities tend to bleach on timescales of minutes resulting in
background variations during the course of the measurement.

\paragraph{Python details}

The estimated background rates are stored in the \verb|Data| attributes
\verb|bg_dd|, \verb|bg_ad| and \verb|bg_aa|, corresponding to photon
streams \verb|Ph_sel(Dex='Dem')|, \verb|Ph_sel(Dex='Aem')| and \verb|Ph_sel(Aex='Aem')|
respectively.
These attributes are lists of arrays (one array per excitation spot).
The arrays contain the estimated background rates in the different time windows
(background periods).
Additional background fitting functions (e.g. least-square fitting of inter-photon delay
histogram) are available in \verb|bg| namespace
(i.e. the \verb|background| module,
\href{http://fretbursts.readthedocs.org/en/latest/background.html}{link}).
