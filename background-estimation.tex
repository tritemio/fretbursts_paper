\subsection{Background Estimation}
\label{sec:bg_calc}

The first step of smFRET analysis involves estimating background rates.
For example, to compute the background every 30 s, using a minimal inter-photon
delay threshold of 2~ms for the photon stream comprising all photons, the corresponding command is:

\begin{lstlisting}
d.calc_bg(bg.exp_fit, time_s=30, tail_min_us=2000)
\end{lstlisting}

The first argument (\verb|bg.exp_fit|) is the function used to fit the
background rate for each photon stream (see section~\ref{sec:bg_intro}).
The function
\verb|bg.exp_fit| estimates the background using a maximum likelihood estimation
(MLE) of the delays distribution.
Additional fitting functions are available in
\verb|bg| namespace 
(i.e. the \verb|background| module, \href{http://fretbursts.readthedocs.org/en/latest/background.html}
{link}). The second argument, \verb|time_s|, is the duration of the
\textit{background period} (section~\ref{sec:bg_intro}) and the third, \verb|tail_min_us|,
is the minimum inter-photon delay to use when fitting the distribution to the specified model function.
It is possible to use different thresholds for each photon stream, passing a
tuple (i.e. a comma-separated list of values, \href{https://docs.python.org/3.5/tutorial/datastructures.html#tuples-and-sequences}{link}) instead of a scalar.
Finally, it is possible to use a heuristic estimation of the threshold using
\verb|tail_min_us='auto'|. For more details refer to \textit{Background estimation} 
section of the μs-ALEX tutorial 
(\href{http://nbviewer.jupyter.org/github/tritemio/FRETBursts_notebooks/blob/master/notebooks/FRETBursts%20-%20us-ALEX%20smFRET%20burst%20analysis.ipynb#Background-estimation}{link}).

FRETBursts provides two kinds of plots to represent the background. One is the histograms
of inter-photon delays compared to the fitted exponential distribution, shown in 
figure~\ref{fig:bg_dist_all}) (see section~\ref{sec:bg_intro} for details on the inter-photon distribution). 
This plot is created with the command:

\begin{lstlisting}
dplot(d, hist_bg, period=0)
\end{lstlisting}

The argument \verb|period| is an integer specifying the background period to be plotted.
When omitted, the default is 0, i.e. the first period is chosen.
Figure~\ref{fig:bg_dist_all} allows to quickly identify pathological cases where the 
background fitting procedure returns unreasonable values. 

The second background-related plot represents a timetrace of background rates, 
as shown in figure~\ref{fig:bg_timetrace}. This plot allows monitoring background rate variationss
occurring during the measurement and is obtained with the command:

\begin{lstlisting}
dplot(d, timetrace_bg)
\end{lstlisting}

Normally, samples should have a fairly constant background rate as a function of time
as in figure~\ref{fig:bg_timetrace}(a). However, sometimes, non-ideal
experimental conditions can yield a time-varying background rate, as illustrated in
figure~\ref{fig:bg_timetrace}(b).
A possible reason for the observed behavior could be buffer evaporation from a poorly sealed (or open) sample observation chamber. Alternatively,
cover-glass impurities can contribute to the background even when focusing 
deep into the sample (10μm or more).
These impurities tend to bleach on timescales of minutes resulting in
background variations during the course of the measurement.
%I don't understand this. 10 um is NOT deep. If anything, you would have to assume increasing adsorption of fluorescent molecules (or impurities) to account for an increase in background rate.
\paragraph{Python details} For ALEX measurements, the tuple passed to
\verb|tail_min_us| in order to define the thresholds needs to contain  
5 values corresponding the 5 distinct photon streams define in section XXX.
%don't remember which section and photon streams. I suggest repeating them here.
The photon streams order can be obtained from
the \verb|Data.ph_streams| attribute (i.e. \verb|d.ph_streams| in our example).
% what is the stream order?
The estimated background rates are stored in the \verb|Data| attributes
\verb|bg_dd|, \verb|bg_ad| and \verb|bg_aa|, corresponding to photon
streams \verb|Ph_sel(Dex='Dem')|, \verb|Ph_sel(Dex='Aem')| and \verb|Ph_sel(Aex='Aem')|
respectively.
% that's 3. Where are the others?
These attributes are lists of arrays (one array per excitation spot).
The arrays contain the estimated background rates in the different time trace windows.

\subsubsection{Error Metrics and Optimal Threshold}
%that sounds like advanced information which might not belong to this introductory paper. If this is discussed in the manual, maybe refer to it?

The functions used to fit the background provide also a goodness-of-fit estimator 
computed on the basis of the empirical distribution function (EDF)~\cite{Stephens1974,Parr1980}. 
The ``distance'' between the EDF and the theoretical (i.e. exponential) cumulative distribution
represents and indicator of the quality of fit.
Two different distance metrics can be returned by the background fitting functions.
The first is the Kolgomorov-Smirnov statistics, which uses the maximum of the difference 
between the EDF and the theoretical distribution. The second is the Cramér von Mises
statistics corresponding to the integral of the squared residuals
(see the code for more details,
\href{https://github.com/tritemio/FRETBursts/blob/master/fretbursts/background.py#L43}{link}).

In principle, the optimal inter-photon delay threshold will minimize
the error metric. This approach is implemented by the function \verb|calc_bg_brute|
(\href{http://fretbursts.readthedocs.org/en/latest/plugins.html#fretbursts.burstlib_ext.calc_bg_brute}{link}) which performs a brute-force search in order to find the optimal threshold.
This optimization is not necessary under typical experimental conditions, 
because the estimated rates normally change only a by a few per-cent 
in most practical cases.
