
\subsection{The \texttt{Data} Class}
\label{sec:data_intro}

The \verb|Data| class
(\href{http://fretbursts.readthedocs.org/en/latest/data_class.html}{link})
is the fundamental data container in FRETBursts. It contains the
measurement data and parameters (attributes) as well as several methods
for data analysis (background estimation, burst search, etc...).
All analysis results (bursts data, estimated parameters) are also stored
as \verb|Data| attributes.

There are 3 important ``burst counts'' attributes which contain
the number of photons detected in the donor or the acceptor channel
during donor or acceptor excitation periods (table~\ref{tab:data_n}).
The attributes in table~\ref{tab:data_n} are background-corrected by default.
Furthermore, \verb|na| is corrected for leakage and direct excitation
(section~\ref{sec:corrcoeff}) if the relative coefficients are specified
(by default they are set to 0).
There is also a closely related attribute named \verb|nda| for donor photons detected
during acceptor excitation. \verb|nda| is normally neglected as it only contains
background.

\begin{table}
\begin{tabular}{l p{0.8\columnwidth}}
  Name  & Description \\
  \hline
  \verb|nd| & number of photons detected by the donor channel (during donor excitation period in ALEX case)\\
  \verb|na| & number of photons detected by the acceptor channel (during donor excitation period in ALEX case)\\
  \verb|naa| & number of photons detected by the acceptor channel during acceptor excitation period (present only in ALEX measurements)\\
\end{tabular}
\caption{\label{tab:data_n}\texttt{Data} attributes names and descriptions for burst photon counts in different photon streams.}
\end{table}


\paragraph{Python details}
Many \verb|Data| attributes are lists of arrays (or scalars) with the length of the lists
equal to the number of excitation spots. This means that in
single-spot measurements, an array of burst-data
is accessed by specifying the index as 0, for example \verb|Data.nd[0]|.
\verb|Data| implements a shortcut syntax to access the first element of a list
with an underscore, so that an equivalently syntax is
\verb|Data.nd_| instead of \verb|Data.nd[0]|.

\subsection{Introduction to Burst Search}
\label{sec:burstsearch_intro}

Identifying single-molecule fluorescence bursts in the stream of photons is
one of the most crucial steps in the analysis of freely-diffusing single-molecule FRET data.
The widely used ``sliding window'' algorithm, introduced by the Seidel group in 
1998~\cite{Eggeling_1998,Fries_1998}, involves searching for
$m$ consecutive photons detected during a period shorter than
$\Delta t$. In other words, bursts are regions of the photon stream where the
local rate (computed using $m$ photons) is above a minimum threshold rate.
Since a universal criterion to choose the rate threshold and
the number of photons $m$ is, as of today, lacking, it has become a common
practice to manually adjust those parameters for each specific measurement.
Commonly employed values for $m$ are between 5 and 15 photons.

A more general approach consists in taking into account the background rate of
the specific measurements and in choosing a rate threshold that is $F$ times
larger than the background rate (typical values for $F$ are between 4 and 9). 
This approach ensures that all resulting bursts
have a signal-to-background ratio (SBR) larger than
$(F-1)$~\cite{Michalet_2012}. A consistent criterion for choosing the threshold is
particularly important when comparing different measurements with different background
rates, when the background significantly varies during measurements or in
multi-spot measurements where each spot has a different background rate.

A second important aspect of burst search is the choice of photon stream used
to perform the search.
In most cases, for instance when identifying FRET sub-populations,
the burst search should use all photons, the so called
all-photon burst search (APBS)~\cite{Eggeling_1998,Fries_1998,Nir_2006}.
In other cases, for example when focusing on
donor-only or acceptor-only populations, it is better to perform 
the search using only donor or acceptor signal.
In order to handle the general case and to provide flexibility,
FRETBursts allows performing the burst search on arbitrary selections of photons
(see section~\ref{sec:ph_streams} for more information on photon stream definitions).

Additionally, Nir~\textit{et al.}~\cite{Nir_2006} proposed a dual-channel 
burst search (DCBS) 
which can help mitigating artifacts due to photophysics effects such as blinking.
During DCBS, a search is performed on two photon streams
and bursts are defined as periods during which both photon streams 
exhibit a rate higher than
the threshold, implementing the equivalent of an AND logic operation.
Conventionally, the term DCBS refers to a burst search where the two photon streams
are (1) all photons during donor excitation (\verb|Ph_sel(Dex='DAem')|) and
(2) acceptor channel photons during acceptor excitation (\verb|Ph_sel(Aex='Aem')|).
In FRETBursts, the user can choose arbitrary photon streams as input, an in general
this kind of search is called a ``AND-gate burst search''.
For additional details on burst search refer to the documentation
(\href{http://fretbursts.readthedocs.io/en/latest/burstsearch.html}{link}).

After burst search, it is necessary to further select
bursts, for instance by specifying a minimum number of photons (or burst size). In the most
basic form, this selection can be performed during burst search by discarding
bursts with size smaller than a threshold $L$ (typically 30 or higher), 
as originally proposed by
Eggeling~\textit{et al.}~\cite{Eggeling_1998}.
This method, however, neglects the effect
of background and $\gamma$ factor on the burst size and can lead to a selection
bias for some channels and/or sub-populations.
For this reason, we suggest performing a burst size selection after background
correction, taking into account the $\gamma$ factor, as discussed in
sections~\ref{sec:burstsizeweights} and~\ref{sec:burstsel}.
In special cases, users may choose to replace (or combine)
the burst selection based on burst size
with another criterion such as burst duration or brightness (see section~\ref{sec:burstsel}).

\subsection{Corrected Burst Sizes and Weights}
\label{sec:burstsizeweights}

The number of photons detected during a burst --the ``burst size''--
is computed using either all photons, or photons detected
during donor excitation period. To compute the burst size, FRETBursts uses
one of the following formulas:

\begin{equation}
\label{eq:burstsize_dex}
n_{dex} = n_a + \gamma\,n_d
\end{equation}

\begin{equation}
\label{eq:burstsize_allph}
n_t = n_a + \gamma\,n_d + n_{aa}
\end{equation}

\noindent where $n_d$, $n_a$ and $n_{aa}$ are, similarly to the attributes
in table~\ref{tab:data_n}, the background-corrected
burst counts in different channels and excitation periods.
The factor $\gamma$ takes into account
different fluorescence quantum yields of donor and acceptor fluorophores and different
photon detection efficiencies between donor and acceptor detection
channels~\cite{Deniz_1999,Lee_2005}.
Eq.~\ref{eq:burstsize_dex} includes counts collected during donor excitation periods only,
while eq.~\ref{eq:burstsize_allph} includes all counts.
Burst sizes computed according to eq.~\ref{eq:burstsize_dex}
or~\ref{eq:burstsize_allph} are called $\gamma$-corrected burst sizes.

The burst search algorithm yields a set of bursts whose sizes
approximately follow an exponential distribution.
Compared to bursts with smaller sizes, bursts with large sizes are less frequent, 
but contain more information per-burst (having higher SNR).
Therefore, selecting bursts by size is an important step (see section~\ref{sec:burstsel}).
A threshold set too low may result in unresolvable sub-populations
because of broadening of FRET peaks and appearance of shot-noise artifacts
in the FRET (and $S$) distribution (i.e. spurious narrow peaks due to $E$ and $S$ being
computed as the ratio of small integers).
Conversely, too large a threshold may result in too low a number of bursts
therefore poor representation of the FRET distribution.
Additionally, especially when computing fractions of sub-populations
(e.g. ratio of number of bursts in each sub-population),
it is important to use $\gamma$-corrected burst sizes as selection criterion,
in order to avoid under-representing some FRET sub-populations
due to different quantum yields of donor and acceptor dyes and/or
different photon detection efficiencies of donor and acceptor channels.

An alternative method to apply the $\gamma$ correction is to
discard a constant fraction of photons chosen randomly from either 
the D\textsubscript{em} or A\textsubscript{em} photon stream~\cite{Nir_2006}. This 
simple method transforms the measurement data in order to
achieve $\gamma=1$, overcoming the issue of selection bias between populations.
This approach has also the advantage of preserving
the binomial distribution of D and A photons in each burst, so that peaks
of FRET populations are easier to model statistically.
The only drawback is that, by discarding a fraction of photons,
this method leads to information loss and therefore to a potential 
decrease in sensitivity and/or accuracy.

A simple way to mitigate the dependence of the FRET distribution on 
the burst size selection threshold is weighting bursts proportionally to their size 
so that the bursts with largest sizes will have the largest weights.
Using size as weights (instead of any other monotonically increasing function
of size) can be justified noticing that the variance of bursts proximity ratio (PR) is
inversely proportional to the burst size (see SI~\ref{sec:burstweights_theory} for details). 

In general, a weighting scheme is used for building efficient estimators for a population
parameter (e.g. the population FRET efficiency $E_p$). 
But, it can also be used to build weighted histograms or Kernel Density
Estimation (KDE) plots which emphasize FRET subpopulations peaks 
without excluding small size bursts.
Traditionally, for optimal results when not using weights, the 
FRET histogram is manually adjusted by finding an ad-hoc (high) 
size-threshold which selects only bursts with the highest size (and thus lowest variance).
Building size-weighted FRET histograms is a simple method to balance 
the need of reducing the peaks width with the need of including as much bursts
as possible to reduce statistical noise.
As a practical example, by fixing the burst size threshold to a low value (e.g. 10-20 photons)
and using weights, is possible to build a FRET histogram with well-defined FRET sub-populations peaks 
without the need of searching an optimal burst-size threshold (SI~\ref{sec:burstweights_theory}).

\paragraph{Python details}
FRETBursts has the option to weight bursts using $\gamma$-corrected
burst sizes which optionally include acceptor excitation photons \verb|naa|.
A weight proportional to the burst size is applied by passing the argument
\verb|weights='size'| to histogram or KDE plot functions. The \verb|weights|
keyword can be also passed to fitting functions in order to fit
the weighted E or S distributions (see section~\ref{sec:fretfit}).
Other weighting functions (for example depending quadratically on the size) 
are listed in the \verb|fret_fit.get_weights| documentation
(\href{http://fretbursts.readthedocs.org/en/latest/fret_fit.html#fretbursts.fret_fit.get_weights}{link}).
However, using weights different from the size is not recommended 
due to their less efficient use of burst information
(SI~\ref{sec:burstweights_theory}).