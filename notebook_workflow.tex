
\subsection{Notebook Workflow}
\label{sec:notebook}

FRETBursts has been developed with the goal of facilitating computational reproducibility
of the performed data analysis~\cite{Buckheit_1995}. For this reason,
the preferential way of using FRETBursts is by executing one of the tutorials
which are in the form of Jupyter notebooks~\cite{Shen_2014}.
Jupyter (formerly IPython) notebooks are web-based documents which contain both
code and rich text (including equations, hyperlinks, figures, etc...).
FRETBursts tutorials are notebooks which can be re-executed,
modified or used to process new data files with minimal modifications.
The ``notebook workflow''~\cite{Shen_2014} not only facilitates
the description of the analysis (by integrating the code in a rich document)
but also greatly enhances its reproducibility by storing an execution trail
that includes software versions, input files, parameters, commands and all
the analysis results (text, figures, tables, etc.).

The Jupyter Notebook environment streamlines FRETBursts execution (compared to
a traditional script and terminal based approach) and allows
FRETBursts to be used even without prior python knowledge.
The user only needs to get familiar with the
notebook graphical environment, in order to be able to navigate and run the notebooks.
A list of all FRETBursts notebooks can be found in the
\verb|FRETBursts_notebooks| repository on GitHub
(\href{https://github.com/tritemio/FRETBursts_notebooks}{link}).
Finally, we provide a service to run FRETBursts notebooks online,
without requiring any software installation 
(\href{https://github.com/tritemio/FRETBursts_notebooks#run-online}{link}).

