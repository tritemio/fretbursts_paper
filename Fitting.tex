\subsection{Population Analysis}
\label{sec:fretfit}

Typically, after bursts selection, E or S histograms are fitted to a model.
FRETBursts \verb|mfit| module allows fitting histograms of bursts quantities
(i.e. E or S) with arbitrary models. In this context, a model is an object
specifying a function, the parameters varied during the fit
and optional constraints for these parameters. This concept of model
is taken from \textit{lmfit}~\cite{lmfit}, the underlying library used by
FRETBursts to perform the fits.

Models can be created from arbitrary functions. By default,
FRETBursts allows using predefined models such as 1 to 3 Gaussian
peaks or 2-Gaussian connected by a ``bridge''.
Built-in models are created by calling a corresponding factory function
(names starting with \verb|mfit.factory_|) which initializes the parameters
with values and constraints suitable for E and S histograms fits.
(see \textit{Factory Functions} documentation,
\href{http://fretbursts.readthedocs.org/en/latest/mfit.html#model-factory-functions}{link}).

As an example, we fit the E histogram of bursts in the
\verb|ds| variable with two Gaussian peaks with the following command:

\begin{lstlisting}
bext.bursts_fitter(ds, 'E', binwidth=0.03,
                   model=mfit.factory_two_gaussians())
\end{lstlisting}

Changing \verb|'E'| with \verb|'S'| will fit the S histogram instead.
The \verb|binwidth| argument specifies the histogram bin width and
the \verb|model| argument defines which model shall be used for
fitting.

All fitting results (including best fit values, uncertainties, etc...),
are stored in the \verb|E_fitter| (or \verb|S_fitter|)
attributes of the \verb|Data| variable (named \verb|ds| here).
To print a comprehensive summary of the fit results, including
uncertainties, reduced $\chi^2$ and correlation between parameters,
the we use the following command:

\begin{lstlisting}
fit_res = ds.E_fitter.fit_res[0]
print(fit_res.fit_report())
\end{lstlisting}

Finally, to plot the fitted model together with the FRET histogram, 
as shown in figure~\ref{fig:histfit}, we pass the parameter \verb|show_model=True|
to the \verb|hist_fret| function as follows
(see section~\ref{sec:plotting} for an introduction to plotting in FRETBursts):

\begin{lstlisting}
dplot(ds, hist_fret, show_model=True)
\end{lstlisting}

For more examples on fitting bursts data and plotting results, refer to the
fitting section of the \usalex notebook (\href{http://nbviewer.jupyter.org/github/tritemio/FRETBursts_notebooks/blob/master/notebooks/FRETBursts%20-%20us-ALEX%20smFRET%20burst%20analysis.ipynb#FRET-fit:-in-depth-example}{link}),
the \textit{Fitting Framework} section of the documentation
(\href{http://fretbursts.readthedocs.org/en/latest/fit.html}{link})
as well as the documentation for \verb|bursts_fitter| function 
(\href{http://fretbursts.readthedocs.org/en/latest/plugins.html#fretbursts.burstlib_ext.bursts_fitter}{link}).

\paragraph{Python details}

Models returned by FRETBursts's factory functions (\verb|mfit.factory_*|)
are \verb|lmfit.Model| objects (\href{https://lmfit.github.io/lmfit-py/model.html}{link}).
Custom models can be created by calling \verb|lmfit.Model| directly.
When an \verb|lmfit.Model| is fitted, it returns a \verb|ModelResults| object
(\href{https://lmfit.github.io/lmfit-py/model.html#the-modelresult-class}{link}),
which contains all information related to the fit (model, data,
parameters with best values and uncertainties) and useful methods to operate on fit results.
FRETBursts puts a \verb|ModelResults| object of each excitation spot in the list
\verb|ds.E_fitter.fit_res|.
For instance, to obtain the reduced $\chi^2$ value of the E histogram fit in a
single-spot measurement \verb|d|, we use the following command:

\begin{lstlisting}
d.E_fitter.fit_res[0].redchi
\end{lstlisting}

Other useful attributes are \verb|aic| and \verb|bic| which contain the
Akaike information criterion (AIC) and the Bayes Information criterion (BIC).
AIC and BIC allow comparing different models and
selecting the most appropriate for the data at hand.

Example of definition and modification of fit models are provided in
the aforementioned \usalex notebook.
Users can also refer to the comprehensive lmfit's documentation
(\href{http://lmfit.github.io/lmfit-py/}{link}).
