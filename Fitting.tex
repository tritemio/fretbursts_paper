\subsection{Population Analysis}
\label{sec:fretfit}

Typically, after bursts selection, E or S histograms are fitted to a model.
FRETBursts \verb|mfit| module allows fitting histograms of bursts quantities
(i.e. E or S) with arbitrary models. In this context, a model is an object
specifying a function, the parameters varied during the fit
and optional constraints for these parameters. This concept of model
is taken from \textit{lmfit}~\cite{lmfit}, the underlying library used by
FRETBursts to perform the fits.

Models can be created from arbitrary functions. 
FRETBursts includes predefined (i.e. built-in) models 
such as 1 to 3 Gaussian peaks or 2-Gaussian connected by a flat plateau.
The latter is an empirical model that
can be used to more accurately fit the center values of two populations
when the peaks are connected by intermediate-FRET bursts
(for the analytical definition of this function see the documentation, 
\href{http://fretbursts.readthedocs.io/en/latest/mfit.html#fretbursts.mfit.factory_two_gaussians}{link}).
Built-in models are created by calling a corresponding factory function
(whose names start with \verb|mfit.factory_|) which initializes the parameters
with values and constraints suitable for E and S histograms fits.
(see \textit{Factory Functions} documentation,
\href{http://fretbursts.readthedocs.org/en/latest/mfit.html#model-factory-functions}{link}).

As an example, we can fit the E histogram of bursts in the
\verb|ds| variable with two Gaussian peaks with the following command:

\begin{lstlisting}
bext.bursts_fitter(ds, 'E', binwidth=0.03,
                   model=mfit.factory_two_gaussians())
\end{lstlisting}

Changing \verb|'E'| with \verb|'S'| will fit the S histogram instead.
The \verb|binwidth| argument specifies the histogram bin width and
the \verb|model| argument defines which model shall be used for
fitting.

All fitting results (including best fit values, uncertainties, etc...),
are stored in the \verb|E_fitter| (or \verb|S_fitter|)
attributes of the \verb|Data| variable (named \verb|ds| here).
To print a comprehensive summary of the fit results, including
uncertainties, reduced $\chi^2$ and correlation between parameters,
the we use the following command:

\begin{lstlisting}
fit_res = ds.E_fitter.fit_res[0]
print(fit_res.fit_report())
\end{lstlisting}

Finally, to plot the fitted model together with the FRET histogram, 
as shown in figure~\ref{fig:histfit}, we pass the parameter \verb|show_model=True|
to the \verb|hist_fret| function as follows
(see section~\ref{sec:plotting} for an introduction to plotting in FRETBursts):

\begin{lstlisting}
dplot(ds, hist_fret, show_model=True)
\end{lstlisting}
