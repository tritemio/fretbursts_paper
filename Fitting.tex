
\subsection{Population fit}

The fitting module \verb|mfit| allows to fit burst populations with a model. Typically, a bursts selection is perfomed first and the histograms of the $E$ or $S$ values of the selected bursts is fitted to a model. Under the hood the
histogram fit is performed using the \href{http://lmfit.github.io/lmfit-py/}{lmfit} library.

In general a model can be automatically built from any function. Before fitting all the model parameters need to have assigned an initial value and (optional) some constrains. A series of functions starting with \verb|factory| in \verb|mfit| return pre-initialized models (for E and S histogram fitting) for commonly used functions, for example 1 or 2 gassian peaks with or without "bridge" function.

For example, to fit the E histogram of bursts in \verb|ds| with two Gaussian peaks we execute:

\begin{verbatim}
bext.bursts_fitter(ds, 'E', binwidth=0.03, model=mfit.factory_two_gaussians())
\end{verbatim}

Here, \verb|ds| is the variable with the burst data (selected bursts), \verb|'E'| is the name of the Data field to fit, \verb|binwidth| is the bin width of the histogram and \verb|model| is a pre-initialized model used for fitting.

After fitting, all the fitting results are stored in the Data variable (in the example in the \verb|E\_fitter| field).
To plot the FRET histogram and the fitted model we run:

\begin{verbatim}
dplot(ds, hist_fret, show_model=True)
\end{verbatim}

For more example on fitting bursts data and plotting results see the \href{http://nbviewer.ipython.org/urls/raw.github.com/tritemio/FRETBursts_notebooks/master/notebooks/FRETBursts\%2520-\%2520us-ALEX\%2520smFRET\%2520burst\%2520analysis.ipynb}{us-ALEX notebook}.


\subsubsection{Correction coefficients}


\begin{itemize}
\item Fit D-only and A-only population.
\item Fit gamma factor.
\end{itemize}


\subsubsection{Accurate FRET}

Apply corrections to the bursts vs apply corrections after the fit.

