
\subsection{Population fit}

The fitting module \verb|mfit| allows to fit burst populations with a model. Typically, a bursts selection is perfomed first and the histograms of the $E$ or $S$ values of the selected bursts is fitted to a model. Under the hood the
histogram fit is performed using the \href{http://lmfit.github.io/lmfit-py/}{lmfit} library.

In general a model can be automatically built from any function. Before fitting all the model parameters need to have assigned an initial value and (optional) some constrains. A series of functions starting with \verb|factory_| in \verb|mfit| return pre-initialized models (for E and S histogram fitting) for commonly used functions, for example 1 or 2 gassian peaks with or without "bridge" function.

For example, to fit the E histogram of bursts in \verb|ds| we execute:

\begin{verbatim}
bext.bursts_fitter(ds, 'E', binwidth=0.03, model=mfit.factory_two_gaussians())
\end{verbatim}

Here, \verb|ds| is the variable with the burst data (selected bursts), \verb|'E'| is the name of the Data field to fit, \verb|binwidth| is the bin width of the histogram and \verb|model| is a pre-initialized model used for fitting.

\item Histogram fit: chose a model, constraints, methods, accuracy
\item KDE: find the maximum
\end{itemize}

\subsubsection{Correction coefficients}


\begin{itemize}
\item Fit D-only and A-only population.
\item Fit gamma factor.
\end{itemize}


\subsubsection{Accurate FRET}

Apply corrections to the bursts vs apply corrections after the fit.

