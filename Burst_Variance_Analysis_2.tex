\subsection{BVA Implementation}

The following paragraphs describe the low-level details involved in implementing the BVA using FRETBursts.
The main goal is to illustrate a real-world example of accessing and manipulating timestamps and burst data.
For a ready-to-use BVA implementation users can refer to the corresponding notebook included with FRETBursts
(\href{http://nbviewer.jupyter.org/github/tritemio/FRETBursts_notebooks/blob/master/notebooks/Example%20-%20Burst%20Variance%20Analysis.ipynb}{link}).

\paragraph{Python details}
For BVA implementation, two photon streams are needed: all-photons during donor excitation (Dex) 
and acceptor photons during donor excitation (DexAem). 
These photon stream selections are obtained by computing boolean masks as follows 
(see section~\ref{sec:burststimes}):

\begin{lstlisting}
Dex_mask = ds.get_ph_mask(ph_sel=Ph_sel(Dex='DAem'))   
DexAem_mask = ds.get_ph_mask(ph_sel=Ph_sel(Dex='Aem'))
DexAem_mask_d = AemDex_mask[Dex_mask]
\end{lstlisting}

Here, the first two variables (\verb|Dex_mask| and \verb|DexAem_mask|)
select photon from the all-photons timestamps array,
while \verb|DexAem_mask_d|, selects A-emitted photons from the
array of photons emitted during D-excitation. As shown below, 
the latter is needed to count acceptor photons in burst chunks.

Next, we need to express bursts start-stop data as indexes of the D-excitation photon stream 
(by default burst start-stop indexes refer to all-photons timestamps array):

\begin{lstlisting}
ph_d = ds_FRET.get_ph_times(ph_sel=Ph_sel(Dex='DAem'))
bursts = ds_FRET.mburst[0] 
bursts_d = bursts.recompute_index_reduce(ph_d)
\end{lstlisting}

Here, \verb|ph_d| contains the Dex timestamps, \verb|bursts| the original burst data and 
\verb|bursts_d| the burst data with start-stop indexes relative to \verb|ph_d|.

Finally, with the previous variables at hand, the BVA algorithm
can be easily implemented by computing the $s_E$ quantity for each burst:

\begin{lstlisting}
n = 7
E_sub_std = []
for burst in bursts_d:
    E_sub = []
    startlist = range(burst.istart, burst.istop + 2 - n, n)
    stoplist = [i + n for i in startlist]
    for start, stop in zip(startlist, stoplist): 
        A_D = DexAem_mask_d[start:stop].sum()
        E = A_D / n
        E_sub.append(E)
    E_sub_std.append(np.std(E_sub))
\end{lstlisting}

Here, \verb|n| is the BVA parameter defining the number of photons in each burst chunk. 
The outer loop iterates through bursts, while the inner loop iterates through sub-bursts.
The variables \verb|startlist| and \verb|stoplist| are the list of start-stop indexes for
all sub-bursts in current burst.
In the inner loop, \verb|A_D| and \verb|E| contain the number of acceptor photons and 
FRET efficiency for the current sub-burst. Finally, for each burst, the standard deviation 
of \verb|E| is appended to the list \verb|E_sub_std|.

By plotting the 2D distribution of $s_E$ (i.e. \verb|E_sub_std|) versus the average (uncorrected) E 
we obtain the BVA plots of figure~\ref{fig:bva_static} and~\ref{fig:bva_dynamic}.

