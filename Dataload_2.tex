
To change the period definitions, we can type:

\begin{lstlisting}
d.add(D_ON=(2100, 3900), A_ON=(100, 1900))
\end{lstlisting}

\noindent where \verb|D_ON| and \verb|A_ON| are pairs of numbers (tuples in Python) representing
the \textit{start} and \textit{stop} values for D or A excitation periods.
The previous command works for both μs-ALEX and ns-ALEX measurements.
After changing the parameters, a new alternation plot will show the updated
period definitions.

The alternation period definition can be applied to the data
using the function \verb|loader.alex_apply_period|
(\href{http://fretbursts.readthedocs.org/en/latest/loader.html#fretbursts.loader.alex_apply_period}{link}):

\begin{lstlisting}
loader.alex_apply_period(d)
\end{lstlisting}

After this command, \verb|d| will contain only photons inside the defined excitation periods.
If the user needs to update the periods definition, the data file will need to be
reloaded and the steps above repeated as described.
