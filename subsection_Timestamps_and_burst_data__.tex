\subsection{Timestamps and burst data}
\label{sec:burststimes}

FRETBursts provides the infrastructure for exploring new analysis approaches.
Users can easily get timestamps (or selection masks) for any photon stream.
Core burst data (essentially start and stop times and indexes 
and derived quantities) are stored in special Bursts objects.
These object provide a simple and well tested interface (100 \% test coverage) 
to access and manipulate burst data. Bursts are created from a sequence of start/stop 
times and indexes, while all the other fields are automatically
computed. Bursts methods allow to recompute indexes relative to a different photon
selection or recompute start and stop times relative to a new timestamps array.
Addiational, methods perform fusion of nearby bursts or intersection of two set
of bursts (functionality used by the dual-channel burst search).
In conclusion, Bursts efficiently implements all the common operations performed 
with burst data, providing and easy-to-use interface and high reliability due to 
its 100 \% unit test coverage. Leveraging Bursts methods, users can implement new 
types of analysis without wasting time implementing (and debugging) standard
manipulation routines.

\begin{itemize}
\item ease of use: fields are accessed by name (e.g. start, stop, counts, etc...)
\item performances: element access and iteration is faster than numpy arrays or Pandas Dataframes
\item pretty printing in the notebook (displayed as HTML table)
\item added functionality: methods to transform burst data to be relative to different timestamps array
\end{itemize}

