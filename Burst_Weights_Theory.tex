\subsection{Theoretical foundation of burst weights}
\label{sec:burstweights_theory}
In homogenuous FRET population, burst counts in the acceptor channel can be
modeled as a binomial random variable with success probability equal to the
population PR and number of trials equal to the burst size $n_d + n_a$.
Similarly, the PR of each burst $E_i$ ($i$ being the burst index) is simply a binomial divided by the 
number of trials, therefore the variance is:

\begin{equation}
\label{eq:E_var}
\operatorname{Var} (E_i) = \frac{E_p\,(1 - E_p)}{n_{ti}}
\end{equation}

For any random variables, a general result is the
Cramer-Rao bound, which states that the Fisher information sets a lower bound on the
variance of any statistics according to eq.~\ref{eq:cramer_rao}
\begin{equation}
\label{eq:cramer_rao}
\operatorname{Var}\left(\hat{p}\right) \ge \frac{1}{\mathcal{I}(\theta)}
\end{equation}

In eq.~\ref{eq:cramer_rao}, $\hat{p}$ is any statistics of the random variable $\theta$,
and $\mathcal{I}(\theta)$ is the Fisher information of $\theta$. When the statistics is
an unbiased estimator of a distribution parameter and the equality holds in eq.~\ref{eq:cramer_rao},
the estimator is a minimum-variance unbiased estimator (MVUB).

we know that any statistics
has a 
sets a lower bound
