\subsection{Burst Weights}
\label{sec:burstweights_theory}
\subsubsection{Theory}
Freely-diffusing molecules across 
a Gaussian excitation volume give rise to 
a burst size distribution that is exponentially distributed.
In a static FRET population, burst counts in the acceptor channel can be
modeled as a binomial random variable (RV) with success probability equal to the
population PR and number of trials equal to the burst size $n_d + n_a$.
Similarly, the PR of each burst $E_i$ ($i$ being the burst index) is 
simply a binomial divided by the number of trials, with variance reported
in eq.~\ref{eq:E_var}.

\begin{equation}
\label{eq:E_var}
\operatorname{Var} (E_i) = \frac{E_p\,(1 - E_p)}{n_{ti}}
\end{equation}

Bursts with higher counts, provide more accurate estimations 
of the population PR, since their PR variance is smaller (eq.~\ref{eq:E_var}). 
Therefore, in estimating the population PR we need to "focus" 
on bigger bursts.
Traditionally, this is accomplished by merely discarding bursts
below a size-threshold.
In the following paragraphs we demonstrate how, by proper weighting
bursts, is possible to obtains optimal estimates of PR which takes into 
account the information of the entire burst population.

According to the Cramer-Rao lower bound (eq.~\ref{eq:cramer_rao}), the 
Fisher information $\mathcal{I}(\theta)$ sets a lower bound on the
variance of any statistics $\hat{p}$ of a RV $\theta$.

\begin{equation}
\label{eq:cramer_rao}
\operatorname{Var}\left(\hat{p}\right) \ge \frac{1}{\mathcal{I}(\theta)}
\end{equation}

When the statistics $\hat{p}$ is an unbiased estimator of a distribution 
parameter and the equality holds in eq.~\ref{eq:cramer_rao},
the estimator is a minimum-variance unbiased (MVUB) estimator 
and it is said to be efficient (meaning that it does an
optimal use the information contained in the sample to estimate the
parameter).

A population of $N$ bursts can be modeled by a set of $N$ binomial
variables with same success probability $E_p$ and varying number of successes
equal to the burst size. An estimator for $E_p$ can be constructed
noticing that the sum of binomial RV with same 
success probability is still a binomial (with number of trials equal to 
the sum of the number of trials).
Taking the sum of acceptor counts over all bursts divided by the total 
number of photons as in eq.~\ref{eq:E_estim}, we obtain 
an estimator $\hat{E}$ of the proportion of successes.

\begin{equation}
\label{eq:E_estim}
\hat{E} = \frac{\sum_i n_{ai}}{\sum_i n_{ti}}
\end{equation}

The variance of $\hat{E}$ (eq.~\ref{eq:E_variance}) is equal to the inverse of 
the Fisher information $\mathcal{I}(\hat{E})$ and therefore $\hat{E}$ is a MVUB 
estimator for $E_p$.

\begin{equation}
\label{eq:E_variance}
\operatorname{Var}(\hat{E}) = \frac{E_p (1 - E_p)}{\sum_i n_{ti}} = \frac{1}{\mathcal{I}(\hat{E})}
\end{equation}

We can finally verify that $\hat{E}$ is equal to the weighted average of the bursts
PR $E_i$ (eq.~\ref{eq:E_wmean}), with weights proportional to the burst size (eq.~\ref{eq:weights}).

\begin{equation}
\label{eq:weights}
w_i
= \frac{n_{ti}}{\sum_i n_{ti}}
\end{equation}

\begin{equation}
\label{eq:E_wmean}
\hat{E}_w = \frac{1}{N} \sum_i w_i E_i 
= \frac{1}{N} \frac{\sum_i n_{ti} \frac{n_{ai}}{n_{ti}} }{\sum_i n_{ti}} = \hat{E}
\end{equation}

Since $\hat{E}$ is the MVUB estimator, any other estimator of $E_p$ (in particular 
the unweighted mean of $E_i$) will have a larger variance.

We can extend these consideration of optimal weights for 
the PR estimator to the FRET distribution plot (histograms or KDEs). Building an
unweighted histogram (and fitting the peak) is analogous to estimating the 
$E_p$ with an unweighted average. Conversely, building the FRET histogram
using the burst size as weights is equivalent to using
the MVUB estimator for $E_p$.

\subsubsection{Weighted FRET estimator}
Here we report a simple verification of the results of previous section, namely
that a weighted mean of $E_i$ is the estimator with minimal variance of $E_p$.
For this purpose, we generated a static FRET population of 100 bursts 
by simply extracting burst-sizes from an exponential distribution ($\lambda = 10$)
and acceptor counts from a binomial distribution ($E_p = 0.2$). 
By repeatedly fitting the population parameter $E_p$ using a 
size-weighted and unweighted average, we verified that the former has systematically
lower variance of the latter as predicted by the theory
(in the current example the unweighted estimator has $28.6\,\%$ higher variance). 
Note that this result
holds for any arbitrary distribution of burst sizes. The full simulation 
including exponential and gamma-distributed burst sizes is reported in
the accompanying Jupyter notebook (\href{http://nbviewer.jupyter.org/github/tritemio/fretbursts_paper/blob/master/notebooks/Figures%20-%20Burst%20Weights.ipynb}{link}).

\subsubsection{Weighted FRET histogram}
The effect of weighting the FRET histogram is here illustrated with a simulation of
a mixture of two static FRET populations and then with experimental data.

We performed a realistic simulation of a static mixture of two FRET populations
starting from 3-D Brownian motion diffusion of $N$ particles excited by a 
numerically computed (non-Gaussian) PSF. Simulation input parameters includes
the diffusion coefficient, the particle brightness, the two FRET efficiencies,
as well as detectors DCR. The simulation is performed with the open source software 
PyBroMo~\cite{Ingargiola_2016} which creates smFRET data files (i.e. timestamps 
and detectors arrays) in Photon-HDF5 format~\cite{Ingargiola2016}.
The simulated data file is processed with FRETBursts performing burst search, 
and only a minimal burst size selection of with threshold of 10 photons.
The resulting weighted and unweighted FRET histograms are reported in figure~\ref{fig:weight_fret_sim}.
We notice that the use of the weights results in better definition of FRET peaks.

As a final comparison, we report the weighted and unweighted FRET histogram of 
an experimental FRET population from measurement of a di-labeled dsDNA sample.
Figure~\ref{fig:weight_fret_meas} show a comparison of a FRET histogram obtained from the same burst
with and without weights. The burst selection is obtained applying a burst size
threshold of 10 counts (after background correction), in order to filter 
the extreme low-end of the burst size distribution.

The use of size-weighted FRET histograms is a simple way to obtain a representation of FRET 
distribution that maintains high power of resolving FRET peaks while including the full burst
population and thus reducing statistical noise.

As a final remark, note that when increasing the size-threshold for burst selection
the difference between weighted and unweighted FRET histograms tends to zero because
the relative difference in burst weights in the selected burst becomes smaller 
(i.e. tends to 1, meaning equal weights).