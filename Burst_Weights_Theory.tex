\subsection{Theoretical foundation of burst weights}
\label{sec:burstweights_theory}
In homogenuous FRET population, burst counts in the acceptor channel can be
modeled as a binomial random variable (RV) with success probability equal to the
population PR and number of trials equal to the burst size $n_d + n_a$.
Similarly, the PR of each burst $E_i$ ($i$ being the burst index) is 
simply a binomial divided by the number of trials, therefore the variance is:

\begin{equation}
\label{eq:E_var}
\operatorname{Var} (E_i) = \frac{E_p\,(1 - E_p)}{n_{ti}}
\end{equation}

For a single FRET population, freely-diffusing molecules across 
a Gaussian excitation volume give rise to 
a burst size distribution that is exponentially distributed.
From eq.~\ref{eq:E_var} we see that bursts with larger burst size
will yield more accurate estimations of the population PR
(since their variance will be smaller). Therefore, in estimating the 
population PR we need to put bigger weights on larger bursts.
In the following paragraphs we demonstrate that the optimal weights
are proportional to the burst size.

According to the Cramer-Rao lower bound (eq.~\ref{eq:cramer_rao}), the 
Fisher information \mathcal{I}(\theta) sets a lower bound on the
variance of any statistics $\hat{p}$ of a RV $theta$.

\begin{equation}
\label{eq:cramer_rao}
\operatorname{Var}\left(\hat{p}\right) \ge \frac{1}{\mathcal{I}(\theta)}
\end{equation}

When the statistics $\hat{p}$ is an unbiased estimator of a distribution 
parameter and the equality holds in eq.~\ref{eq:cramer_rao},
the estimator is a minimum-variance unbiased estimator (MVUB)
and the estimator is said to be efficient (meaning that it does an
optimal use the information contained in the sample to estimate the
parameter).

A population of $N$ bursts can be modeled by a set of $N$ binomial
variables with same success probability $E_p$ and varying number of successes
equal to the burst size. Now, notice that the sum of binomial RV with same 
success probability is still a binomial (with number of trial equal to 
the sum of the number of trials).
Therefore we can build an estimator for $E_p$ by computing the proportion of
success from the sum over all bursts of acceptor counts o divided by the total
number of photons as in eq.~\ref{eq:E_estim}.

\begin{equation}
\label{eq:E_estim}
\hat{E} = \frac{\sum_i n_{ai}}{\sum_i n_{ti}}
\end{equation}

The variance of $\hat{E}$ (eq.~\ref{eq:E_var}) is equal to the inverse of 
the Fisher information $\mathcal{I}(\hat{E})$ and therefore $\hat{E}$ is a MVUB 
estimator for $E_p$.

\begin{equation}
\label{eq:E_var}
\operatorname{Var}(\hat{E}) = \frac{E_p\,(1 - E_p)}{\sum_i n_{ti}} = \frac{1}{\mathcal{I}(\hat{E})
\end{equation}

With simple algebra is verify that $hat{E}$ is equal to the weighted average of the bursts
PR $E_i$ (eq.~ref{}), with weights proportional to the burst size.

\begin{equation}
\label{eq:weights}
w_i = \frac{\frac{n_{ti}}{p(1-p)}}{\frac{1}{p(1-p)} \sum_i n_{ti}}
= \frac{n_{ti}}{\sum_i n_{ti}}
\end{equation}

\begin{equation}
\label{eq:E_wmean}
\hat{E}_w = \frac{1}{N} \sum_i w_i E_i = \frac{1}{N}  \frac{\sum_i n_{ti} E_i}{\sum_i n_{ti}}
= \frac{1}{N}  \frac{\sum_i n_{ti} \frac{n_{ai}}{n_{ti}} }{\sum_i n_{ti}}
= \frac{1}{N}  \frac{\sum_i n_{ai}}{\sum_i n_{ti}} = \hat{E}
\end{equation}



