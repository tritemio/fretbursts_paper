It is advisable to monitor the background as a function of time
throughout the measurement, in order to account for possible variations.
Experimentally, we found that when the background is not constant,
it usually varies
on time scales of tens of seconds (see figure~\ref{fig:bg_timetrace}).
FRETBursts divides the acquisition in constant-duration time
windows called \textit{background periods} and computes the background rates for
each of these windows (see section~\ref{sec:bg_calc}).
Note that FRETBursts uses these local background rates also during burst search, 
in order to compute time-dependent burst detection thresholds 
and for background correction of burst data (see section~\ref{sec:burstsearch}).
