Single-molecule FRET (smFRET) allows probing intermolecular interactions and conformational changes in biomacromolecules, and represent an invaluable tool in studying cellular processes on the molecular scale\cite{Kapanidis_2006}. smFRET experiments can detect the distance between two fluorescent labels (\textit{donor} and \textit{acceptor}) in the 3-10 nm range. In the commonly employed confocal geometry, molecules are free to diffuse in a solution. When a molecule traverses the excitation volume it emits a burst of photons that can be detected by single-photon detectors (SPADs). 
The relative intensities of signals from donor and acceptor fluorescence are related to the distance between the two dyes.

smFRET experiments are sometimes considered complex to properly analyze. The analysis of smFRET experiments involves identifying photon bursts from single-molecules in a continuous stream of photon timestamps, estimating the background and other correction factors, filtering and extracting the corrected FRET efficiencies for each sub-population in the sample. In order to ease the process, we herein present FRETBursts, a new software for smFRET data analysis that includes most of the common state-of-the-art algorithms. We follow the highest standard in software development to ensure that the free source is easy to read, well documented and thoroughly tested. Moreover, in an effort to lower the barriers to computational reproducibility, we embrace a modern workflow based on ipython notebooks that allows capture of the whole process from raw data to figures within a single document.
