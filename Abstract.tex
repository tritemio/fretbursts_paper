Single-molecule Förster Resonance Energy Transfer (smFRET) allows probing intermolecular interactions
and conformational changes in biomacromolecules, and represents an 
invaluable tool in studying cellular processes at the molecular 
scale~\cite{Kapanidis_2006}. smFRET experiments can detect the distance between 
two fluorescent labels (\textit{donor} and \textit{acceptor}) in the 
3-10~nm range. In the commonly employed confocal geometry, molecules are free 
to diffuse in solution. When a molecule traverses the excitation volume it 
emits a burst of photons that can be detected by single-photon avalanche detectors (SPADs). 
The intensities of donor and acceptor fluorescence 
can then be related to the distance between the two dyes.

The analysis of smFRET experiments involves identifying photon bursts from 
single-molecules in a continuous stream of photon, estimating the 
background and other correction factors, filtering and finally extracting the corrected 
FRET efficiencies for each sub-population in the sample. In order to simplify 
the process, we have developed FRETBursts, an open source software for confocal smFRET data 
analysis that includes most of the common state-of-the-art algorithms. 
We follow the highest standard in software development to ensure that 
the source is easy to read, well documented and thoroughly tested. 
Moreover, in an effort to lower the barriers to computational reproducibility, 
we embrace a modern workflow based on Jypyter notebooks that allows to capture 
of the whole process from raw data to figures within a single document.
