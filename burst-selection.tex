\subsection{Burst Selection}
\label{sec:burstsel}

After burst search, it is common to select bursts according to different
criteria. One of the most common is burst size.

For instance, to select bursts with more than 30 photons detected during the donor excitation 
(computed after background correction), the following command is used:

\begin{lstlisting}
ds = d.select_bursts(select_bursts.size, th1=30)
\end{lstlisting}

The previous command creates a new \textit{Data} variable (\verb|ds|) containing
the selected bursts. \verb|th1| defines the lower bound for burst size, while 
\verb|th2| (not shown here) defines the upper bound (when not specified 
the upper bound is $+\infty$).

As mentioned before, the new object will share the photon data
arrays with the original object (\verb|d|) in order to minimize the amount of memory used to store these objects.

The first argument of \verb|select_bursts|
(\href{http://fretbursts.readthedocs.org/en/latest/data_class.html#burst-selection-methods}{link})
is a python function implementing the "selection rule" (\verb|select_bursts.size| in this example);
all remaining arguments (only \verb|th1| in this case) are parameters of the selection rule.
The \verb|select_bursts| module
(\href{http://fretbursts.readthedocs.org/en/latest/burst_selection.html}{link})
contains numerous built-in selection functions
(\href{http://fretbursts.readthedocs.org/en/latest/burst_selection.html#module-fretbursts.select_bursts}{link}).
For example,
\verb|select_bursts.ES|
is used to select a region on the E-S ALEX histogram,
\verb|select_bursts.width|
to select bursts based on their duration.
New criteria can be easily implemented by defining a new selection function,
which requires as few as a couple of lines of code in most cases (see the
\verb|select_bursts| module's source code
for several examples,
\href{https://github.com/tritemio/FRETBursts/blob/master/fretbursts/select_bursts.py}{link}).

Finally, different criteria can be combined sequentially.
For example, the following commands:

\begin{lstlisting}
ds = d.select_bursts(select_bursts.size,
                     th1=50, th2=200)
dsw = ds.select_bursts(select_bursts.width,
                       th1=0.5e-3, th2=3e-3)
\end{lstlisting}

apply both a burst size and a burst duration selection criterion, in which bursts
have sizes between 50 and 200 photons, and duration between 0.5 and 3~ms.

\subsubsection{Burst Size Selection}

In the previous section, we selected bursts by size, using only photons
detected in the  donor and acceptor channels during donor excitation.
Alternatively, a threshold on the total burst size (including all photons) can be applied
(section~\ref{sec:burstsizeweights}) by adding $n_{aa}$ to the burst size
as in eq.~\ref{eq:burstsize_allph}. This is achieved
by passing \verb|add_naa=True| to the selection function. When \verb|add_naa| is not specified,
as in the previous section, the default \verb|add_naa=False| is used
(i.e. use only photons during donor excitation). The complete selection command
is:

\begin{lstlisting}
ds = d.select_bursts(select_bursts.size,
                     th1=30, add_naa=True)
\end{lstlisting}

\noindent The result of this selection is plotted in figure~\ref{fig:alex_jointplot}.

Another important parameter for defining the burst size is the γ-factor, i.e.
the imbalance between the donor and the acceptor channel signals. As noted in
section~\ref{sec:burstsizeweights}, the γ-factor is
used to compensate bias for the different fluorescence quantum yields of the D and A
fluorophores as well as the different photon-detection efficiencies of the D and A channels.
When γ is significantly different from 1, neglecting its effect on burst size leads to
over-representing (in terms of number of bursts) one population versus the others.

When the γ factor is known, the argument
\verb|gamma| can be passed during burst selection:

\begin{lstlisting}
ds = d.select_bursts(select_bursts.size,
                     th1=15, gamma=0.65)
\end{lstlisting}

When not specified, $\gamma=1$ is assumed.

For more information on burst size selection, see the
\verb|select_bursts.size| documentation
(\href{http://fretbursts.readthedocs.org/en/latest/burst_selection.html#fretbursts.select_bursts.size}{link}).

\paragraph{Python details} The method \verb|Data.burst_sizes|
(\href{http://fretbursts.readthedocs.org/en/latest/data_class.html#fretbursts.burstlib.Data.burst_sizes}{link})
computes and returns γ-corrected burst sizes with or without addition of \verb|naa|.

\subsubsection{Select the FRET Populations}

In smFRET-ALEX experiments, in addition to one or more FRET populations, there are always
donor-only (D-only) and acceptor-only (A-only) populations.
In most cases, these additional populations are not of interest and need to be filtered out.

In principle, using the E-S representation, D-only and A-only bursts
can be excluded by selecting bursts within a range of $S$ values (e.g. S=0.2-0.8). 
This approach, however, simply truncates the burst distribution with arbitrary
thresholds and is therefore not recommended for quantitative assessment of FRET
populations.
% I don't think that is a very strong statement, in the sense that it is just as ad hoc as the E-S selection method
% and leaves dubious intermediate bursts in the final result

An alternative approach consists in applying two selection filters sequentially.
First, the A-only population is filtered out
%some FRET burst are also filtered out this way
by applying a threshold on the number of photons during donor excitation.
Second, the D-only population is fitlered out by applying a threshold on the number of photons during acceptor excitation.
%same comment as previously
The commands for these combined selections are:

\begin{lstlisting}
ds1 = d.select_bursts(select_bursts.size, th1=15)
ds2 = ds1.select_bursts(select_bursts.naa, th1=15)
\end{lstlisting}

Here, variable \verb|ds2| contains the combined burst selection.
Figure~\ref{fig:alex_jointplot_fretsel} shows the resulting pure FRET
population obtained with the previous selection.
