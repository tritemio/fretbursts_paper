\subsection{Burst Selection}
\label{sec:burstsel}

After burst search, it is common to select bursts according to different
criteria. One of the most common is burst size.

For instance, to select bursts with more than 30 photons detected during the donor excitation (computed
after background correction), the following command is used:

\begin{lstlisting}
ds = d.select_bursts(select_bursts.size, th1=30)
\end{lstlisting}

The previous command creates a new \textit{Data} variable (\verb|ds|) containing
the selected bursts.
As mentioned before the new object will share the photon data
arrays with the original object (\verb|d|) in order to minimize the RAM use.

The first argument of \verb|select_bursts|
(\href{http://fretbursts.readthedocs.org/en/latest/data_class.html#burst-selection-methods}{link})
is a python function implementing the "selection rule" (\verb|select_bursts.size| in this example);
all the remaining arguments (only \verb|th1| in this case) are parameters of the selection rule.
The \verb|select_bursts| module
(\href{http://fretbursts.readthedocs.org/en/latest/burst_selection.html}{link})
contains numerous built-in selection functions
(\href{http://fretbursts.readthedocs.org/en/latest/burst_selection.html#module-fretbursts.select_bursts}{link}).
For example,
\verb|select_bursts.ES|
is used to select a region on the E-S ALEX histogram,
\verb|select_bursts.width|
to select bursts based on their duration.
New criteria can be easily implemented by defining a new selection function,
which requires not more than a couple of lines of code in most cases (see the
\verb|select_bursts| module's source code
for several examples,
\href{https://github.com/tritemio/FRETBursts/blob/master/fretbursts/select_bursts.py}{link}).

Finally, different criteria can be combined by applying them sequentially.
For example, with the following commands:

\begin{lstlisting}
ds = d.select_bursts(select_bursts.size,
                     th1=50, th2=200)
dsw = ds.select_bursts(select_bursts.width,
                       th1=0.5e-3, th2=3e-3)
\end{lstlisting}

we apply a combined burst size and duration selection, in which bursts
have sizes between 50 and 200 photons, and duration between 0.5 and 3~ms.

\subsubsection{γ-corrected Burst Size Selection}

In the previous section, we selected bursts by size using only photons
detected by donor and acceptor channel during donor excitation.
Conversely, we can apply a threshold on the all-photon burst size
(section~\ref{sec:burstsizeweights}) by adding $n_{aa}$ to the burst size
as in eq.~\ref{eq:burstsize_allph}. This is achieved
by passing \verb|add_naa=True| to the selection function. When \verb|add_naa| is not specified,
as in the previous section, the default \verb|add_naa=False| is used
(i.e. use only photons during donor excitation). The complete selection command
is:

\begin{lstlisting}
ds = d.select_bursts(select_bursts.size,
                     th1=30, add_naa=True)
\end{lstlisting}

\noindent and the resulting selection is plotted in figure~\ref{fig:alex_jointplot}.

Another important parameter for defining the burst size is the γ-factor, i.e.
the imbalance between the donor and the acceptor channels. As noted in
section~\ref{sec:burstsizeweights}, the γ-factor is
used to compensate bias for the different fluorescence quantum yields of the D and A
fluorophores as well as the different photon-detection efficiencies of the D and A channels.
When γ-factor is not 1, neglecting its effect on burst size leads to
over-representing (in terms of number of bursts) one population versus to the others.

When the γ factor is known, users can pass the argument
\verb|gamma| during burst selection:

\begin{lstlisting}
ds = d.select_bursts(select_bursts.size,
                     th1=15, gamma=0.65)
\end{lstlisting}

When not specified, $\gamma=1$ is assumed.

For more information on burst size selection refer to the
\verb|select_bursts.size| documentation
(\href{http://fretbursts.readthedocs.org/en/latest/burst_selection.html#fretbursts.select_bursts.size}{link}).

\paragraph{Python details} The method \verb|Data.burst_sizes|
(\href{http://fretbursts.readthedocs.org/en/latest/data_class.html#fretbursts.burstlib.Data.burst_sizes}{link})
computes and returns γ-corrected burst sizes with or without addition of \verb|naa|.

\subsubsection{Select the FRET Populations}

In smFRET-ALEX experiments, in addition to one or more FRET populations, there are always
donor-only (D-only) and acceptor-only (A-only) populations.
In most cases, these additional populations are not of interest and need to be filtered out.

In principle, using the E-S representation, we can exclude D-only and A-only bursts
by selecting bursts within a range of $S$ values (e.g. S=0.2-0.8). 
This approach, however, simply truncates the burst distribution with arbitrary
thresholds and is therefore not recommended for quantitative assessment of FRET
populations.

A better approach consists in applying two selection filters one after the other.
First, we filter out the A-only population
by applying a threshold on number of photons during donor excitation.
Second, we exclude the D-only population by
by applying a threshold on number of photons during acceptor excitation.
The commands for this combined selections are:

\begin{lstlisting}
ds1 = d.select_bursts(select_bursts.size, th1=15)
ds2 = ds1.select_bursts(select_bursts.naa, th1=15)
\end{lstlisting}

Here, the variable \verb|ds2| contains the composite selection of bursts.
Figure~\ref{fig:alex_jointplot_fretsel} shows the resulting pure FRET
population obtained with the previous selection.
