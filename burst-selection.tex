\subsection{Burst Selection}
\label{sec:burstsel}

After burst search, it is common to select bursts according to different
criteria. One of the most common is burst size.

For instance, to select bursts with more than 30 photons detected during the donor excitation 
(computed after background correction), we use following command:

\begin{lstlisting}
ds = d.select_bursts(select_bursts.size, th1=30)
\end{lstlisting}

The previous command creates a new \verb|Data| variable (\verb|ds|) containing
the selected bursts. \verb|th1| defines the lower bound for burst size, while 
\verb|th2| defines the upper bound (when not specified, as in the previous example,
the upper bound is $+\infty$).
As before, the new object (\verb|ds|) will share the photon data
arrays with the original object (\verb|d|) in order to minimize the amount 
of used memory.

The first argument of \verb|select_bursts|
(\href{http://fretbursts.readthedocs.org/en/latest/data_class.html#burst-selection-methods}{link})
is a python function implementing the ``selection rule'' (\verb|select_bursts.size| in this example);
all remaining arguments (only \verb|th1| in this case) are parameters of the selection rule.
The \verb|select_bursts| module
(\href{http://fretbursts.readthedocs.org/en/latest/burst_selection.html}{link})
contains numerous built-in selection functions
(\href{http://fretbursts.readthedocs.org/en/latest/burst_selection.html#module-fretbursts.select_bursts}{link}).
For example,
\verb|select_bursts.ES|
is used to select a region on the E-S ALEX histogram,
\verb|select_bursts.width|
to select bursts based on their duration.
New custom criteria can be readily implemented by defining a new selection function,
which requires only a couple of lines of code (see the
\verb|select_bursts| module's source code for examples,
\href{https://github.com/tritemio/FRETBursts/blob/master/fretbursts/select_bursts.py}{link}).

Finally, different criteria can be combined sequentially.
For example, with the following commands:

\begin{lstlisting}
ds = d.select_bursts(select_bursts.size,
                     th1=50, th2=200)
dsw = ds.select_bursts(select_bursts.width,
                       th1=0.5e-3, th2=3e-3)
\end{lstlisting}

bursts in \verb|dsw|
will have sizes between 50 and 200 photons, and duration between 0.5 and 3~ms.

\paragraph{Burst Size Selection}
In the previous section, we selected bursts by size, using only 
photons detected in both D and A channels during D excitation (i.e. Dex photons), 
as in eq.~\ref{eq:burstsize_dex}.
Alternatively, a threshold on the burst size computed including all photons 
can be applied by adding $n_{aa}$ to the burst size (see eq.~\ref{eq:burstsize_allph}). 
This is achieved
by passing \verb|add_naa=True| to the selection function. 
The complete selection command is:

\begin{lstlisting}
ds = d.select_bursts(select_bursts.size,
                     th1=30, add_naa=True)
\end{lstlisting}

\noindent The result of this selection is plotted in figure~\ref{fig:alex_jointplot}.
When \verb|add_naa| is not specified,
as in the previous section, the default is \verb|add_naa=False|
(i.e. compute size using only Dex photons). 