\section{smFRET Burst Analysis}
\label{sec:analysis}

\subsection{Loading the Data}
\label{sec:dataload}
While FRETBursts can load several data files formats,
we encourage users to adopt the recently introduced Photon-HDF5 file format~\cite{Ingargiola2016}.
Photon-HDF5 is an HDF5-based, open format, specifically designed for freely-diffusing smFRET and
other timestamp-based experiments.
Photon-HDF5 is a self-documented, platform- and language-independent binary format,
which supports compression and allows saving photon data (e.g. timestamps) and measurement-specific metadata
(e.g. setup and sample information, authors, provenance, etc.).
Moreover, Photon-HDF5 is designed for long-term data preservation and aims to facilitate data sharing
between different software and research groups.
All example data files provided with FRETBursts use the Photon-HDF5 format.

To load data from a Photon-HDF5 file, we use the function \verb|loader.photon_hdf5|
(\href{http://fretbursts.readthedocs.org/en/latest/loader.html#fretbursts.loader.photon_hdf5}{link}):

\begin{lstlisting}
d = loader.photon_hdf5(filename)
\end{lstlisting}

\noindent
where \verb|filename| is a string containing the file path.
This command loads the measurement data into the variable \verb|d|,
a \verb|Data| object (see section~\ref{sec:data_intro}).

The same command can load data from a variety of smFRET measurements supported
by the Photon-HDF5 format, taking advantage of the rich metadata included with each file.
For instance, data generated using different excitation schemes such as CW excitation
or pulsed excitation, single-laser vs two alternating lasers, etc.,
or with any number of excitation spots, are automatically recognized and interpreted accordingly.

FRETBursts also supports loading \usalex data stored in .sm files
(a custom binary format used in the Weiss lab),
ns-ALEX data stored in .spc files (a binary format used by TCSPC Becker \& Hickl acquisition hardware).
Alternatively, these and other formats (such as ht3, a binary format used by PicoQuant hardware)
can be converted into Photon-HDF5 files using phconvert,
a file conversion library and utility for Photon-HDF5
(\href{http://photon-hdf5.github.io/phconvert/}{link}).
More information on loading different file formats
can be found in the \verb|loader| module's documentation
(\href{http://fretbursts.readthedocs.org/en/latest/loader.html}{link}).

\subsection{Alternation Parameters}
\label{sec:alternation}

For \usalex and ns-ALEX data, Photon-HDF5 normally stores parameters defining
alternation periods corresponding to donor and acceptor laser excitation.
At load time, a user can plot these parameters and change them if deemed necessary.
In \usalex measurements~\cite{Kapanidis_2004},
CW laser lines are alternated on timescales of the order of 10 to 100~\us.
Plotting an histogram of timestamps modulo the alternation period, it
is possible to identify the donor and acceptor excitation periods (see figure~\ref{fig:altern_hist_double}a).
In ns-ALEX measurements~\cite{Laurence_2005},
pulsed lasers with equal repetition rates are delayed with respect
to one another with typical delays of 10 to 100~ns.
In this case, forming an histogram of TCSPC times (nanotimes) will allow
the definition of periods of fluorescence after excitation
of either the donor or the acceptor (see figure~\ref{fig:altern_hist_double}b).
In both cases, the function
\verb|plot_alternation_hist|
(\href{http://fretbursts.readthedocs.org/en/latest/plots.html#fretbursts.burst_plot.plot_alternation_hist}{link})
will plot the relevant alternation histogram (figure~\ref{fig:altern_hist_double})
using currently selected (or default) values for donor and acceptor excitation periods.
