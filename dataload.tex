\section{smFRET burst analysis}
\label{sec:analysis}

\subsection{Loading the data}
\label{sec:dataload}
\lstset{language=C}
\begin{lstlisting}
#include <stdio.h>
#define N 10
/* Block
 * comment */

int main()
{
    int i;

    // Line comment.
    puts("Hello world!");
    
    for (i = 0; i < N; i++)
    {
        puts("LaTeX is also great for programmers!");
    }

    return 0;
}
\end{lstlisting}
While FRETBursts can load data files from a few file formats,
the authors promote \href{http://photon-hdf5.readthedocs.org/}{Photon-HDF5},
an HDF5-based open format specifically designed for freely-diffusing smFRET and other timestamp-based experiments.
Photon-HDF5 is a self-documented platform and language independent binary format
which support compression and allows saving photon-data (e.g. timestamps) and measurement-specific meta-data
(setup and sample information, authors, provenance etc...).
Moreover, Photon-HDF5 is designed for long-term data preservation and aims to facilitate data sharing
among different software and research groups.
FRETBursts example data files are in Photon-HDF5 format and can be opened with
stand-alone viewers (such as \href{http://www.hdfgroup.org/products/java/hdfview/}{HDFView}) or 
programming environment.
Additional data files can be found here[multi-spot paper data files].

To load data from a Photon-HDF5 file we use the
\href{http://fretbursts.readthedocs.org/en/latest/loader.html#fretbursts.loader.photon_hdf5}{function \texttt{loader.photon\_hdf5}} as follows:

\begin{lstlisting}
d = loader.photon_hdf5(file_name)
\end{lstlisting}

\noindent
where \verb|file_name| is a string containing the file path.
This command loads all the measurement data into the variable \verb|d|,
a \href{http://fretbursts.readthedocs.org/en/latest/data_class.html}{\texttt{Data}} object
(see section~\ref{sec:data_intro}).

The same command can load data from a variety of measurement types stored
in a Photon-HDF5 file. For instance, data generated using different excitation schemes
(CW vs pulsed, single-laser vs 2 alternating lasers) or with any number of excitation spots
is automatically recognized.

Other file formats which FRETBursts can load include μs-ALEX data stored in sm format
(a binary format formerly used in our lab),
ns-ALEX data stored in SPC format (a binary format used by TCSPC Becker \& Hickl cards).
More information on loading these file formats and on manually loading other arbitrary formats
can be found in the
\href{http://fretbursts.readthedocs.org/en/latest/loader.html}{\texttt{loader} module documentation}.

\subsection{Alternation parameters}
\label{sec:alternation}

In case of µs-ALEX and ns-ALEX data, it is necessary to define the
alternation periods for donor and acceptor excitation. 
In µs-ALEX measurements, CW lasers are alternated on timescales of 10-100~µs.
By plotting the histogram of the timestamps modulo the alternation period
is possible to identify the donor and acceptor periods (see figure~\ref{fig:altern_hist_double}a).
In ns-ALEX measurements, pulsed lasers are interleaved with typical separation of 10-100~ns.
In this case the histogram of the TCSP nanotimes will allow to define the period
starting with a donor (or acceptor) excitation pulse (see figure~\ref{fig:altern_hist_double}b).

In both cases, the functions
\verb|plot_alternation_hist| (\href{http://fretbursts.readthedocs.org/en/latest/plots.html#fretbursts.burst\_plot.plot\_alternation\_hist}{documentation})
will plots the relevant alternation histogram (figure~\ref{fig:altern_hist_double}) 
using currently selected (or default) values for donor and acceptor excitation periods.

To change the period definitions, the user can type:

\begin{verbatim}
    d.add(D_ON=(2850, 580), A_ON=(900, 2580))
\end{verbatim}

where \verb|D_ON| and \verb|A_ON| are tuples (pairs of numbers) representing
the \textit{start} and \textit{stop} values for D or A excitation periods.
The previous command works both for µs-ALEX and ns-ALEX measurements.

After changing the parameters, a new alternation plot will show the updated 
period selections.

When the alternation period definition is correctly defined, it can
be applied using the
\href{http://fretbursts.readthedocs.org/en/latest/loader.html#fretbursts.loader.alex_apply_period}{function \texttt{alex\_apply\_period}}:

\begin{verbatim}
    loader.alex_apply_period(d)
\end{verbatim}

After this command, \verb|d| will contain only photons inside the defined excitation periods. 
At this point, in order to further change the period definitions,
it is necessary to reload the data file.
