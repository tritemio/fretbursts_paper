\section{smFRET Burst Analysis}
\label{sec:analysis}

\subsection{Loading the Data}
\label{sec:dataload}
While FRETBursts can load several data files formats,
we encourage users to adopt the recently introduced Photon-HDF5 file format~\cite{Ingargiola2016}.
Photon-HDF5 is an HDF5-based, open format, specifically designed for freely-diffusing smFRET and 
other timestamp-based experiments.
Photon-HDF5 is a self-documented, platform- and language-independent binary format,
which supports compression and allows saving photon data (e.g. timestamps) and measurement-specific metadata
(e.g. setup and sample information, authors, provenance, etc.).
Moreover, Photon-HDF5 is designed for long-term data preservation and aims to facilitate data sharing
between different software and research groups.
All example data files provided with FRETBursts use the Photon-HDF5 format.

To load data from a Photon-HDF5 file, the function \verb|loader.photon_hdf5|
(\href{http://fretbursts.readthedocs.org/en/latest/loader.html#fretbursts.loader.photon_hdf5}{link})
is used:

\begin{lstlisting}
d = loader.photon_hdf5(filename)
\end{lstlisting}

\noindent
where \verb|filename| is a string containing the file path.
This command loads all measurement data into the variable \verb|d|, 
a \verb|Data| object (see section~\ref{sec:data_intro}).

This command can load data from a variety of smFRET measurements supported
by the Photon-HDF5 format, taking advantage of the rich metadata included with each file.
For instance, data generated using different excitation schemes such as CW excitation or pulsed excitation, single-laser vs two alternating lasers, etc., or with any number of excitation spots, are automatically recognized and interpreted accordingly.

FRETBursts also supports loading μs-ALEX data stored in .sm files
(a custom binary format used in the Weiss lab),
ns-ALEX data stored in .spc files (a binary format used by TCSPC Becker \& Hickl data acquisition software) and
ns-ALEX data in .ht3 files (a binary format used by PicoQuant data acquisition software).
Alternatively, these files can be converted into Photon-HDF5 files using phconvert, a file conversion utility developed to facilitate the adoption of Photon-HDF5 (\href{http://photon-hdf5.github.io/phconvert/}{link}), prior to their processing with FRETBursts.
More information on loading these and other file formats 
can be found in the \verb|loader| module's documentation
(\href{http://fretbursts.readthedocs.org/en/latest/loader.html}{link}).

\subsection{Alternation Parameters}
\label{sec:alternation}

For µs-ALEX and ns-ALEX data, Photon-HDF5 normally stores parameters defining
alternation periods corresponding to donor and acceptor laser excitation.
At load time, a user can plot these parameters and change them if deemed necessary.
In µs-ALEX measurements~\cite{Kapanidis_2004},
CW laser lines are alternated on timescales of the order of 10 to 100~µs.
Plotting an histogram of timestamps modulo the alternation period, it
is possible to identify the donor and acceptor excitation periods (see figure~\ref{fig:altern_hist_double}a).
In ns-ALEX measurements~\cite{Laurence_2005},
pulsed lasers with equal repetition rates are delayed with respect to one another with typical delays of 10 to 100~ns.
In this case, forming an histogram of TCSPC times (nanotimes)
will allow 
the definition of the period of fluorescence after excitation of either the donor or the acceptor
(see figure~\ref{fig:altern_hist_double}b).
%this could be rephrased, since there could be several excitation periods within a TAC window.
In both cases, the function
\verb|plot_alternation_hist| 
(\href{http://fretbursts.readthedocs.org/en/latest/plots.html#fretbursts.burst_plot.plot_alternation_hist}{link})
will plot the relevant alternation histogram (figure~\ref{fig:altern_hist_double}) 
using currently selected (or default) values for donor and acceptor excitation periods.

To change the period definitions, the user can type:

\begin{lstlisting}
d.add(D_ON=(2100, 3900), A_ON=(100, 1900))
\end{lstlisting}

where \verb|D_ON| and \verb|A_ON| are tuples (pairs of numbers) representing
the \textit{start} and \textit{stop} values for D or A excitation periods.
The previous command works for both µs-ALEX and ns-ALEX measurements.
%give an example with multiple alternation periods per TAC  window?
After changing the parameters, a new alternation plot will show the updated 
period definitions.

When the alternation period definition is specified, it can
be used using the function \verb|loader.alex_apply_period|
(\href{http://fretbursts.readthedocs.org/en/latest/loader.html#fretbursts.loader.alex_apply_period}{link}):

\begin{lstlisting}
loader.alex_apply_period(d)
\end{lstlisting}

After this command, \verb|d| will contain only photons inside the defined excitation periods.
If the user needs to update the periods definition, the data file will need to be reloaded and the steps above repeated as described.
