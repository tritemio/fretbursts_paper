
Another important parameter for defining the burst size is the $\gamma$-factor, i.e.
the imbalance between the donor and the acceptor channel signals. As noted in
section~\ref{sec:burstsizeweights}, the $\gamma$-factor is
used to compensate bias for the different fluorescence quantum yields of the D and A
fluorophores as well as the different photon-detection efficiencies of the D and A channels.
When $\gamma$ is significantly different from 1, neglecting its effect on burst size leads to
over-representing (in terms of number of bursts) one FRET population versus the others.

When the $\gamma$ factor is known, a more unbiased selection of different FRET
populations can be achieved passing the argument \verb|gamma| to the 
selection function:

\begin{lstlisting}
ds = d.select_bursts(select_bursts.size,
                     th1=15, gamma=0.65)
\end{lstlisting}

When not specified, $\gamma=1$ is assumed.

For more details on burst size selection, see the
\verb|select_bursts.size| documentation
(\href{http://fretbursts.readthedocs.org/en/latest/burst_selection.html#fretbursts.select_bursts.size}{link}).

\paragraph{Python details} 
To compute $\gamma$-corrected burst sizes (with or without addition of \verb|naa|) 
the method \verb|Data.burst_sizes|
(\href{http://fretbursts.readthedocs.org/en/latest/data_class.html#fretbursts.burstlib.Data.burst_sizes}{link})
is used.

\subsubsection{Select the FRET Populations}

In smFRET-ALEX experiments, in addition to one or more FRET populations, there are always
donor-only (D-only) and acceptor-only (A-only) populations.
In most cases, these additional populations are not of interest and need to be filtered out.

In principle, using the E-S representation, D-only and A-only bursts
can be excluded by selecting bursts within a range of $S$ values (e.g. S=0.2-0.8). 
This approach, however, simply truncates the burst distribution with arbitrary
thresholds and is therefore not recommended for quantitative assessment of FRET
populations.

An alternative approach consists in applying two selection filters sequentially.
First, the A-only population is filtered out
by applying a threshold on the number of photons during D excitation.
Second, the D-only population is filtered out by applying a threshold on 
the number of photons during A excitation.
The commands for these combined selections are:

\begin{lstlisting}
ds1 = d.select_bursts(select_bursts.size, th1=15)
ds2 = ds1.select_bursts(select_bursts.naa, th1=15)
\end{lstlisting}

Here, variable \verb|ds2| contains the combined burst selection.
Figure~\ref{fig:alex_jointplot_fretsel} shows the resulting pure FRET
population obtained with the previous selection.
