\subsection{Background estimation}

The first step of smFRET analysis involves estimating the background level. In fact due to detectors dark counting rates (DCR), samples scattering and autofluorescence there is a always a minimum level of counts even when no molecule is crossing the excitation volume and emitting photons. Figure X shows the typical distribution of timestamps delays (waiting times between two subsequent timestamps). The "tail" of the distribution (a line in semi-log scale) correspond to exponetially-distributed delays, indicating that we have Poisson proccess generating background counts.

\begin{itemize}
\item Background as a function of time
\end{itemize}

\subsubsection{Choice of the threshold}
\begin{itemize}
\item heuristic
\item brute force
\end{itemize}

\subsection{Burst search}

Description of burst-search algorithms and why the m-photons sliding windows is exactly the same as fixed-time sliding window. Maybe a picture will help.

\begin{itemize}
\item Adaptive threshold as a function of background
\item Chosing different photon streams
\item AND-Gate
\end{itemize}

\subsection{Burst selection}

How to select bursts according to different criteria (size, width, E, S, topN, etc...).

How to define a new criterium.


\subsection{Population fit}

\begin{itemize}
\item Histogram fit: chose a model, constraints, methods, accuracy
\item KDE: find the maximum
\end{itemize}

\subsubsection{Correction coefficients}

\begin{itemize}
\item Fit D-only and A-only population.
\item Fit gamma factor.
\end{itemize}


\subsubsection{Accurate FRET}

Apply corrections to the bursts vs apply corrections after the fit.

