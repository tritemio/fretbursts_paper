\section{FRETBursts Overview}
\label{sec:overview}

\subsection{Technical Features}

FRETBursts can analyze smFRET measurements
from one or multiple excitation spots~\cite{Ingargiola_2013}. The supported
excitation schemes include single laser, alternating laser excitation (ALEX)
with either CW lasers (μs-ALEX~\cite{Kapanidis_2005})
or pulsed lasers (ns-ALEX~\cite{Laurence_2005} or
pulsed-interleaved excitation (PIE)~\cite{M_ller_2005}).

The software implements both standard and novel algorithms for smFRET data analysis
including background estimation as a function of time (including background accuracy
metrics), sliding-window burst search~\cite{Eggeling_1998},
dual-channel burst search (DCBS)~\cite{Nir_2006} and
modular burst selection methods based on user-defined criteria
(including a large set of pre-defined selection rules). Novel features include burst size
selection with $\gamma$-corrected burst sizes, burst weighting, burst search with
background-dependent threshold (in order to guarantee a minimal signal-to-background
ratio~\cite{Michalet_2012}).
Moreover, FRETBursts provides a large set of fitting options to characterize FRET subpopulations.
In particular, distributions of burst quantities (such as $E$ or $S$) can be assessed
through (1) histogram fitting (with arbitrary model functions),
(2) non-parametric weighted kernel density estimation (KDE), (3) weighted
expectation-maximization (EM), (4) maximum likelihood fitting using Gaussian models
or Poisson statistic. Finally FRETBursts includes a large number of
predefined and customizable plot functions which (thanks to the \textit{matplotlib}
graphic library) produce publication quality plots in a wide range of formats.

Additionally, implementations of population dynamics analysis such
as Burst Variance Analysis (BVA)~\cite{Torella_2011} and two-channel
kernel density distribution estimator (2CDE)~\cite{Tomov_2012}
are available as FRETBursts notebooks.

\subsection{Software Availability}
FRETBursts is hosted and openly developed on GitHub. FRETBursts homepage
(\href{http://tritemio.github.io/FRETBursts}{link})
contains links to the various resources. Installation instructions can be found in the
Reference Documentation (\href{http://fretbursts.readthedocs.org/en/latest/getting_started.html}{link}).
A description of FRETBursts execution using Jupyter notebooks is reported
in SI~\ref{sec:notebook}.
Detailed information on development style, testing strategies and
contributions guidelines are reported in SI~\ref{sec:dev}.
Finally, to facilitate evaluation and comparison with other software,
we set up an on-line services allowing to execute FRETBursts
without requiring any installation on the user's computer (\href{https://github.com/tritemio/FRETBursts_notebooks#run-online}{link}).
