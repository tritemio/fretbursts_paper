\section{Architecture and Concepts}
\label{sec:concepts}

In this section, we introduce some general concepts and notations used in FRETBursts.

\subsection{Photon Streams}
\label{sec:ph_streams}

The raw data collected during a smFRET experiment consists in one or more arrays of 
photon timestamps, whose temporal resolution is set by the acquisition hardware, 
typically between 10 and 50 ns.
In single-spot measurements, all timestamps are stored in a single array. In multispot
measurements~\cite{Ingargiola_2013}, there are as many timestamps arrays
as excitation spots.

Each array contains timestamps from both donor (D) and acceptor (A) channels.
When alternating excitation lasers are used (ALEX measurements)~\cite{Lee_2005}, 
a further distinction between photons emitted during the D or A excitation periods can be made. 
In FRETBursts, the corresponding sets of photons are called ``photon streams'' and are
specified with a \verb|Ph_sel| object
(\href{http://fretbursts.readthedocs.org/en/latest/ph_sel.html}{link}).
In non-ALEX smFRET data, there are 3 photon streams
(table~\ref{tab:ph_sel_smfret}), while in ALEX data, there are 5 streams (table~\ref{tab:ph_sel_alex}).

The \verb|Ph_sel| class (\href{http://fretbursts.readthedocs.org/en/latest/ph_sel.html}{link})
allows the specification of any combination of photon streams.
For example, in ALEX measurements, the D-emission during A-excitation stream is
usually ignored because it does not contain any useful signal~\cite{Lee_2005}.
To indicate all but photons in this photon stream, the syntax is
\verb|Ph_sel(Dex='DAem', Aex='Aem')|, which indicates selection of donor
and acceptor photons (\verb|DAem|) during donor excitation (\verb|Dex|) and only acceptor
photons (\verb|Aem|) during acceptor excitation (\verb|Aex|).

\begin{table}
\begin{tabular}{l|l}
  Photon selection  & code \\
  \hline
  All-photons       & \verb|Ph_sel('all')|\\
  D-emission    & \verb|Ph_sel(Dex='Dem')|\\
  A-emission & \verb|Ph_sel(Dex='Aem')|\\
\end{tabular}
\caption{\label{tab:ph_sel_smfret}Photon selection syntax (non-ALEX)}
\end{table}

\begin{table}
\begin{tabular}{l|l}
  Photon selection  & code \\
  \hline
  All-photons & \verb|Ph_sel('all')|\\
  D-emission during D-excitation & \verb|Ph_sel(Dex='Dem')|\\
  A-emission during D-excitation & \verb|Ph_sel(Dex='Aem')|\\
  D-emission during A-excitation & \verb|Ph_sel(Aex='Dem')|\\
  A-emission during A-excitation & \verb|Ph_sel(Aex='Aem')|\\
\end{tabular}
\caption{\label{tab:ph_sel_alex}Photon selection syntax (ALEX)}
\end{table}

\subsection{Background Definitions}
\label{sec:bg_intro}

An estimation of the background rates is needed to both select a proper threshold for
burst search, and correct the raw burst counts by subtraction of background counts.

The recorded stream of timestamps is the result of two processes: one characterized
by a high count rate, due to fluorescence photons of single molecules crossing the
excitation volume, and another characterized by a lower count rate, due to "background
counts" originating from detector dark counts, afterpulsing, out-of-focus molecules
and sample scattering and/or impurities~\cite{Edman_1996,Gopich_2008}.
The signature of these two types of processes can be
observed in the inter-photon delays distribution (i.e. the waiting times
between two subsequent timestamps) as illustrated in figure~\ref{fig:bg_dist_all}(a).
The “tail” of the distribution (a straight line in semi-log scale) corresponds
to exponentially-distributed time-delays, indicating that those counts are generated by a
Poisson process. At short
timescales, the distribution departs from the exponential due to the contribution
of the higher rate process of single molecules traversing the excitation volume.
To estimate the background rate (i.e. the inverse of the exponential time constant),
it is necessary to define a time-delay threshold above which the distribution
can be considered exponential.
Finally, a parameter estimation method needs to be specified, such as Maximum
Likelihood Estimation (MLE) or non-linear least squares (NLLS) curve fitting of 
the time-delay histogram (both supported in FRETBursts).

It is advisable to monitor the background as a function of time
throughout the measurement, in order to account for possible variations.
Experimentally, we found that when the background is not constant,
it usually varies
on time scales of tens of seconds (see figure~\ref{fig:bg_timetrace}).
FRETBursts divides the acquisition in constant-duration time
windows called \textit{background periods} and computes the background rates for
each of these windows (see section~\ref{sec:bg_calc}).
Note that FRETBursts uses these local background rates also during burst search, 
in order to compute time-dependent burst detection thresholds 
and for background correction of burst data (see section~\ref{sec:burstsearch}).

\subsection{The \texttt{Data} Class}
\label{sec:data_intro}

The \verb|Data| class
(\href{http://fretbursts.readthedocs.org/en/latest/data_class.html}{link})
is the fundamental data container in FRETBursts. It contains the
measurement data and parameters (attributes) as well as several methods
for data analysis (background estimation, burst search, etc...).
All analysis results (bursts data, estimated parameters) are also stored
as \verb|Data| attributes.

There are 3 important ``burst counts'' attributes which contain
the number of photons detected in the donor or the acceptor channel
during donor or acceptor excitation (table~\ref{tab:data_n}).
The attributes in table~\ref{tab:data_n} are background-corrected by default.
Furthermore, \verb|na| is corrected for leakage and direct excitation
(section~\ref{sec:corrcoeff}) if the relative coefficients are specified
(by default they are 0).
There is also a closely related attribute named \verb|nda| for donor photons
during acceptor excitation. \verb|nda| is normally neglected as it only contains
background.

\begin{table}
\begin{tabular}{l p{0.8\columnwidth}}
  Name  & Description \\
  \hline
  \verb|nd| & number of photons detected by the donor channel (during donor excitation period in ALEX case)\\
  \verb|na| & number of photons detected by the acceptor channel (during donor excitation period in ALEX case)\\
  \verb|naa| & number of photons detected by the acceptor channel during acceptor excitation period (present only in ALEX measurements)\\
\end{tabular}
\caption{\label{tab:data_n}\texttt{Data} attributes names and descriptions for burst photon counts in different photon streams.}
\end{table}


\paragraph{Python details}
Many \verb|Data| attributes are lists of arrays (or scalars) with the length of the lists
equal to the number of excitation spots. This means that in
single-spot measurements, an array of burst-data
is accessed by specifying the index as 0, for example \verb|Data.nd[0]|.
\verb|Data| implements a shortcut syntax to access the first element of a list
with an underscore, so that an equivalently syntax is
\verb|Data.nd_| instead of \verb|Data.nd[0]|.

\subsection{Introduction to Burst Search}
\label{sec:burstsearch_intro}

Identifying single-molecule fluorescence bursts in the stream of photons is
one of the most crucial steps in the analysis of freely-diffusing single-molecule FRET data.
The widely used ``sliding window'' algorithm, introduced by the Seidel group in 1998
(\cite{Eggeling_1998}, \cite{Fries_1998}), involves searching for
$m$ consecutive photons detected during a period shorter than
$\Delta t$. In other words, bursts are regions of the photon stream where the
local rate (computed using $m$ photons) is above a minimum threshold rate.
Since a universal criterion to choose the rate threshold and
the number of photons $m$ is, as of today, lacking, it has become a common
practice to manually adjust those parameters for each specific measurement.

A more general approach consists in taking into account the background rate of
the specific measurements and in choosing a rate threshold that is $F$ times
larger than the background rate. This approach ensures that all resulting bursts
have a signal-to-background ratio (SBR) larger than
$(F-1)$~\cite{Michalet_2012}. A consistent criterion for choosing the threshold is
particularly important when comparing different measurements with different background
rates, when the background significantly varies during measurements or in
multi-spot measurements where each spot has a different background rate.

A second important aspect of burst search is the choice of photon stream used
to perform the search.
In most cases, for instance when identifying FRET sub-populations,
the burst search should use all photons (i.e. APBS). In some other cases, when focusing on
donor-only or acceptor only populations, it is better to perform the search using
only donor or acceptor signal.
In order to handle the general case and to provide flexibility,
FRETBursts allows performing the burst search on arbitrary selections of photons.
(see section~\ref{sec:ph_streams} for more information on photon stream definitions).

Additionally, Nir~\textit{et al.}~\cite{Nir_2006} proposed DCBS ('dual-channel burst search'), which can help mitigating artifacts due to photophysics effects such as blinking.
During DCBS, a search is performed in parallel on two photon streams
and bursts are defined as periods during which both photon streams exhibit a rate higher than
the threshold, implementing the equivalent of an AND logic operation.
Conventionally, the term DCBS refers to a burst search where the two photon streams
are (1) all photons during donor excitation (\verb|Ph_sel(Dex='DAem')|) and
(2) acceptor channel photons during acceptor excitation (\verb|Ph_sel(Aex='Aem')|).
In FRETBursts, the user can choose arbitrary photon streams as input, an in general
this kind of search is called a 'AND-gate burst search'.

After burst search, it is necessary to select
bursts, for instance by specifying a minimum number of photons (or burst size). In the most
basic form, this selection can be performed during burst search by discarding
bursts with size smaller than a threshold $L$, as originally proposed by
Eggeling~\textit{et al.}~\cite{Eggeling_1998}.
This method, however, neglects the effect
of background and γ factor on the burst size and can lead to a selection
bias for some channels and/or sub-populations.
For this reason, we suggest performing a burst size selection after background
correction, taking into account the γ factor, as discussed in
sections~\ref{sec:burstsizeweights} and~\ref{sec:burstsel}.
In special cases, users may choose to replace (or combine)
the burst selection based on burst size
with another criterion such as burst duration or brightness (see section~\ref{sec:burstsel}).

\subsection{$\gamma$-corrected Burst Sizes and Weights}
\label{sec:burstsizeweights}

The number of photons detected during a burst --the ``burst size''--
is computed using either all photons, or photons detected
during donor excitation period. To compute the burst size, FRETBursts uses
one of the following formulas:

\begin{equation}
\label{eq:burstsize_dex}
n_{dex} = n_a + \gamma\,n_d
\end{equation}

\begin{equation}
\label{eq:burstsize_allph}
n_t = n_a + \gamma\,n_d + n_{aa}
\end{equation}

\noindent where $n_d$, $n_a$ and $n_{aa}$ are, similarly to the attributes
in table~\ref{tab:data_n}, the background-corrected
burst counts in different channels and excitation periods.
The factor γ takes into account
different fluorescence quantum yields of donor and acceptor fluorophores and different
photon detection efficiencies between donor and acceptor detection
channels~\cite{Deniz_1999,Lee_2005}.
Eq.~\ref{eq:burstsize_dex} includes counts collected during donor excitation periods only,
while eq.~\ref{eq:burstsize_allph} includes all counts.
Burst sizes computed according to eq.~\ref{eq:burstsize_dex}
or~\ref{eq:burstsize_allph} are called γ-corrected burst sizes.

The burst search algorithm yields a set of bursts whose sizes
approximately follow an exponential distribution.
Compared to bursts with smaller sizes, bursts with large sizes are less frequent, 
but contain more information per-burst (having higher SNR).
Therefore, selecting bursts by size is an important step (see section~\ref{sec:burstsel}).
A threshold set too low may result in unresolvable sub-populations
because of broadening of FRET peaks and appearance of shot-noise artifacts
in the FRET (and S) distribution (i.e. spurious narrow peaks due to E and S being
computed as the ratio of small integers).
Conversely, too large a threshold may result in too low a number of bursts
therefore poor statistical significance of the computed FRET distribution.
Additionally, especially when computing fractions of sub-population 
(e.g. ratio of number of bursts in each sub-population),
it is important to use γ-corrected burst sizes,
in order to avoid under-representing some FRET sub-populations
due to different quantum yields of donor and acceptor dyes and/or
different photon detection efficiencies of donor and acceptor channels.

A simple way to mitigate the dependence of the FRET distribution on 
the burst size selection threshold is weighting bursts proportionally to their size 
(i.e. their Fisher information)
so that the bursts with largest sizes will have the largest weights.
In statistics, weights proportional to the inverse of the variance are used 
in the context of least-squares fitting to take into account samples
with different variances. In the case of a burst population with fixed FRET efficiency $E_p$, 
the counts $n_a$ are commonly described by binomial random variable, with $n_d + n_a$ trials 
and $E_p$ probability of success. While this model neglects the effect of background, it captures
the main features of a static FRET population. Under the binomial assumption,
the variance of each burst $E_i$ is inversely proportional to the burst size $n_{ti}$ 
(where $i$ is the index of the burst),
as reported in eq.~\ref{eq:var_e}.

\begin{equation}
\label{eq:var_e}
\operatorname{Var}\left( E_i \right)
= \frac{E_p(1-E_p)}{n_{ti}}
\end{equation}

Therefore weights proportional to $n_{ti}$ represent a natural choice (see SI~\ref{sec:burstweights_theory}).
In general, such a weighting scheme is used for building efficient estimators for a population
parameter (e.g. $E_p$). But, it can also be used to build weighted histograms or Kernel Density
Estimation (KDE) plots which exhibit FRET subpopulations with minimal width,
yielding more accurate fit of peaks positions and better resolution of nearby peaks
(compared to corresponding non-weighted plots using the same burst-size threshold).
Traditionally, for optimal results when not using weights, the width of 
FRET subpopulations peaks are manually optimized by finding an ad-hoc (high) 
size-threshold which selects only bursts with the highest size (and thus lowest variance).
This procedure, needs to balance the reduction in peaks width due to selection of bigger bursts
with the increase in statistical noise due to the reduction in number of bursts.
Conversely, the use of size weights allows a more efficient use of burst information.
For example, by fixing the burst size threshold to a low value (e.g. 20 photons) is possible 
to obtain optimal-width FRET sub-populations peaks without any need to search
for an optimal burst-size threshold (see SI~\ref{sec:burstweights_theory}).

\paragraph{Python details}
FRETBursts has the option to weight bursts using γ-corrected
burst sizes which optionally include acceptor excitation photons \verb|naa|.
A weight proportional to the burst size is applied by passing the argument
\verb|weights='size'| to histogram or KDE plot functions. The \verb|weights|
keyword can be also passed to fitting functions in order to fit
the weighted E or S distributions (see section~\ref{sec:fretfit}).
Other weighting functions (for example depending quadratically on the size) 
are listed in the \verb|fret_fit.get_weights| documentation
(\href{http://fretbursts.readthedocs.org/en/latest/fret_fit.html#fretbursts.fret_fit.get_weights}{link}).
However, using weights different from the size is not recommended 
due to their less efficient use of burst information.