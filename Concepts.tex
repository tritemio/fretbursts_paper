\section{Architecture and concepts}
\label{sec:concepts}

In this section we introduce some general concepts and naming conventions related 
to the smFRET burst analysis in FRETBursts.

\subsection{Photon streams}
\label{sec:ph_streams}

The fundamental data at the core of smFRET experiments is the array of photon
arrival timestamps, with a resolution of the order of 10~ns. In single-spot
measurements, all the timestamps are stored in a single array. In multi-spot
measurements we have as many timestamps arrays as the number of excitation
spots.

Each array contains timestamps from both the donor (D) and the
acceptor (A) channels. In ALEX measurements~\cite{Lee_2005}, we can further 
differentiate between
photons emitted during D and A excitation periods. In FRETBursts the different
selections of photons/timestamps are called "photon streams" and they are
specified with a
\href{http://fretbursts.readthedocs.org/en/latest/ph_sel.html}{\texttt{Ph\_sel}
object} . In non-ALEX smFRET data we have 3 base photon streams
(table~\ref{tab:ph_sel_smfret}), while in ALEX data we have 5 base photon
streams (table~\ref{tab:ph_sel_alex}).

The
\href{http://fretbursts.readthedocs.org/en/latest/ph_sel.html}{\texttt{Ph\_sel}
class} allows the expression of any combination of photon streams. 
For example, in ALEX measurements, the D-emission during A-excitation stream is
usually excluded because it does not contain any useful signal~\cite{Lee_2005}.
To indicate all but the photons in this photon stream we write
\verb|Ph_sel(Dex='DAem', Aex='Aem')|, which literally means \textit{select donor
and acceptor photons (DAem) during donor excitation (Dex) and only acceptor
photons (Aem) during acceptor excitation (Aex)}.

\begin{table}
\begin{tabular}{l|l}
  Photon selection  & code \\
  \hline
  All-photons       & \verb|Ph_sel('all')|\\
  D-emission    & \verb|Ph_sel(Dex='Dem')|\\
  A-emission & \verb|Ph_sel(Dex='Aem')|\\
\end{tabular}
\caption{\label{tab:ph_sel_smfret}Photon selection syntax (non-ALEX)}
\end{table}

\begin{table}
\begin{tabular}{l|l}
  Photon selection  & code \\
  \hline
  All-photons & \verb|Ph_sel('all')|\\
  D-emission during D-excitation & \verb|Ph_sel(Dex='Dem')|\\
  A-emission during D-excitation & \verb|Ph_sel(Dex='Aem')|\\
  D-emission during A-excitation & \verb|Ph_sel(Aex='Dem')|\\
  A-emission during A-excitation & \verb|Ph_sel(Aex='Aem')|\\
\end{tabular}
\caption{\label{tab:ph_sel_alex}Photon selection syntax (ALEX)}
\end{table}

\subsection{Background definitions}
\label{sec:bg_intro}

The recorded stream of timestamps is comprised of two processes: one with 
high rate, due to fluorescence photons of single molecules crossing the 
excitation volume, and another with lower rate due to “background
counts” originating from the detectors dark counts, out of focus molecules
and sample scattering and/or auto-fluorescence\cite{Gopich_2008}. 
These two processes can be 
observed in the distribution of timestamps delays (i.e. the waiting times 
between two subsequent timestamps) as illustrated in Figure~\ref{fig:bgdist}.
The “tail” of the distribution (a line in semi-log scale) corresponds 
to exponentially-distributed delays, indicating that those counts are generated by a
\href{http://en.wikipedia.org/wiki/Poisson_process}{Poisson process}. At short
timescales, the distribution departs from the exponential due to the bursts of
photons from diffusing single-molecules. To estimate the background
rate, (i.e. the exponential time constant) we need to select a minimal
timestamp delay threshold, above which the distribution can be considered exponential. We also need to choose a fitting method, for example the Maximum
Likelihood Estimation (MLE) or a curve fit of the histogram via non-linear
least squares (NLSQ).

Both burst search and burst correction require knowledge of background rates for all the
different photon streams. Furthermore, we want to estimate the background
periodically (every few seconds) because it can vary during the measurement on
time scales of tens of seconds. FRETBursts splits the data in uniform time
slices called \textit{background periods} and computes the background rates for
each of these slices (see section~\ref{sec:bg_calc}). Note that the slicing 
in background
periods is also used during burst search to compute a background-dependent
threshold and to apply the burst correction (section~\ref{sec:burstsearch}).

\subsection{The \texttt{Data} class}
\label{sec:data_intro}

The
\href{http://fretbursts.readthedocs.org/en/latest/data_class.html}{\texttt{Data}
class} is the fundamental data container in FRETBursts. It contains the
measurement data and provides several methods for data analysis (background
estimation, burst search, etc...). It also stores all the analysis results
(bursts data, estimated parameters).

All the arrays in \texttt{Data} are contained in lists whose length is equal to the
number of excitation spots. This means that for single-spot measurements all the
arrays are wrapped in 1-element lists. For example, the bursts data field
\verb|Data.mbursts| will be a 1-element list and \verb|Data.mbursts[0]| will be
the actual numpy array of burst data. \verb|Data|implements a shortcut syntax
that allows accessing 
\verb|Data.mbursts[0]| as \verb|Data.mbursts_| (valid for all the fields).

As an example the following are some important burst-data fields:

\begin{itemize}
\item \verb|nd|: number of photons detected through the donor channel (during
donor excitation), after correction
\item \verb|na|: number of photons detected through the acceptor channel (during
donor excitation), after correction
\item \verb|naa|: number of photons detected through the acceptor channel during
acceptor excitation, after correction
\end{itemize}

\subsection{Plotting "Data"}

FRETBursts uses matplotlib~\cite{2096e2a4-8f50-4519-bfb3-f796da201630} to
provide a wide range of 
\href{http://fretbursts.readthedocs.org/en/latest/plots.html}{built-in plot functions}
for \verb|Data| objects. 
The plot syntax is the same both for single and multi-spot measurements. 
Almost all the plot commands are called through the wrapper function 
\verb|dplot|, for example to plot a timetrace of the photon data we type:

\begin{verbatim}
dplot(d, timetrace)
\end{verbatim}

The function \verb|dplot| is the generic plot function that creates the figure
and handles details common to all the plotting functions (i.e. the title).
\verb|d| is the \verb|Data| variable and \verb|timetrace| is the actual plot
function that operates on a single channel. In multi-spot measurements
\verb|dplot| creates one subplot for each spot and calls \verb|timetrace| for
each channel.

All the built-in plot functions that can be passed to 
\verb|dplot| are defined in the 
\href{http://fretbursts.readthedocs.org/en/latest/plots.html}{\texttt{burst\_plot} module}. 
When importing FRETbursts all the plot functions are also imported. 
To facilitate finding the plot functions through auto-completion, 
their names start with a standard prefix indicating the
plot type. The prefixes are: \verb|timetrace| for binned timetraces
of photon data, \verb|ratetrace| for rates of photons as a function of time (non
binnned), \verb|hist| for functions plotting histograms and \verb|scatter| for
scatter plots.

Additional plots can be easily created directly with matplotlib.

Usually plots are displayed inline in the notebook. However a few plot functions
 such as \verb|timetrace| and \verb|hist2d_alex| have interactive features that
can be enabled when using the QT4 backend that opens the plot in an external
window. It is possible to switch backend from inline to QT and vice versa using
the ipython commands \verb|%matplotlib qt|
and \verb|%matplotlib inline|. For example, after switching to the QT4 backend
the following commads:

\begin{verbatim}
dplot(d, timetrace, scroll=True)
\end{verbatim}

opens a new window with a timetrace plot and an horizontal scrollbar for quick
"scrolling" throughout the measurement.
Similarly, calling the \verb|hist2d_alex| function with the QT4 backend allows
selecting an area on the E-S histogram using the mouse.

\begin{verbatim}
dplot(ds, hist2d_alex, gui_sel=True)
\end{verbatim}

The values that identify the region are printed and can be copied an pasted as
argument for the burst selection function \verb|select_bursts.ES| (see
section~\ref{sec:burstsel}).

\subsection{Burst weights}
As will be shown in the following sections, the burst search returns a 
burst population with burst of different sizes (i.e. number of photons). 
The bursts with largest size (that contains indeed the most information) 
are much less frequent than bursts with small size. For this reason, in order 
to resolve FRET populations and accurately fit their FRET efficiency it is of
paramount importance to select bursts with size larger than a given threshold (see 
section~\ref{sec:burstsel}). The choice of the burst size threshold is 
critical since a too low threshold will widen the FRET peaks because of 
the inclusion of small sized bursts having higher FRET uncertainty, conversely a too high threshold results in a lower number of bursts and FRET estimation can suffer from high statitical error.

In order to mitigate the dependence on the burst size threshold it is 
useful to weight bursts according to their size (i.e. their 
information content) so that the biggest bursts will have the highest weights. 
The weighting can be used to build weighted histograms and Kernel Density Estimation (KDE) plots.
When using weights the choice of a particular burst size threshold becomes 
much less important and the histograms or KDE plots exhibit sharp
peaks corresponding to different FRET popuations for a much wider range
of thresholds. 

FRETBursts allows weighting bursts using the gamma-corrected burst size
(\verb|gamma*nd + na|) and optionally adding the acceptor counts during
acceptor excitation (\verb|naa|). Once the size is defined, the weights
can be computed according to different criteria among which the most
commonly used is \verb|'size'|, meaning to associate a weight 
directly proportional to the burst size. The list of weigth types
can be found in the 
\href{http://fretbursts.readthedocs.org/en/latest/fret_fit.html#fretbursts.fret_fit.get_weights}{\texttt{fret\_fit.get\_weiths} documentation}.

