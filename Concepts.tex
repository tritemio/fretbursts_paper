\section{Architecture and Concepts}
\label{sec:concepts}

In this section, we introduce some general concepts and notations used in FRETBursts.

\subsection{Photon Streams}
\label{sec:ph_streams}

The raw data collected during a smFRET experiment consists in one or more arrays of 
photon timestamps, whose temporal resolution is set by the acquisition hardware, 
typically between 10 and 50 ns.
In single-spot measurements, all timestamps are stored in a single array. In multispot
measurements~\cite{Ingargiola_2013}, there are as many timestamps arrays
as excitation spots.

Each array contains timestamps from both donor (D) and acceptor (A) channels.
When alternating excitation lasers are used (ALEX measurements)~\cite{Lee_2005}, 
a further distinction between photons emitted during the D or A excitation periods can be made. 
In FRETBursts, the corresponding sets of photons are called ``photon streams'' and are
specified with a \verb|Ph_sel| object
(\href{http://fretbursts.readthedocs.org/en/latest/ph_sel.html}{link}).
In non-ALEX smFRET data, there are 3 photon streams
(table~\ref{tab:ph_sel_smfret}), while in ALEX data, there are 5 streams (table~\ref{tab:ph_sel_alex}).

The \verb|Ph_sel| class (\href{http://fretbursts.readthedocs.org/en/latest/ph_sel.html}{link})
allows the specification of any combination of photon streams.
For example, in ALEX measurements, the D-emission during A-excitation stream is
usually ignored because it does not contain any useful signal~\cite{Lee_2005}.
To indicate all but photons in this photon stream, the syntax is
\verb|Ph_sel(Dex='DAem', Aex='Aem')|, which indicates selection of donor
and acceptor photons (\verb|DAem|) during donor excitation (\verb|Dex|) and only acceptor
photons (\verb|Aem|) during acceptor excitation (\verb|Aex|).

\begin{table}
\begin{tabular}{l|l}
  Photon selection  & code \\
  \hline
  All-photons       & \verb|Ph_sel('all')|\\
  D-emission    & \verb|Ph_sel(Dex='Dem')|\\
  A-emission & \verb|Ph_sel(Dex='Aem')|\\
\end{tabular}
\caption{\label{tab:ph_sel_smfret}Photon selection syntax (non-ALEX)}
\end{table}

\begin{table}
\begin{tabular}{l|l}
  Photon selection  & code \\
  \hline
  All-photons & \verb|Ph_sel('all')|\\
  D-emission during D-excitation & \verb|Ph_sel(Dex='Dem')|\\
  A-emission during D-excitation & \verb|Ph_sel(Dex='Aem')|\\
  D-emission during A-excitation & \verb|Ph_sel(Aex='Dem')|\\
  A-emission during A-excitation & \verb|Ph_sel(Aex='Aem')|\\
\end{tabular}
\caption{\label{tab:ph_sel_alex}Photon selection syntax (ALEX)}
\end{table}

\subsection{Background Definitions}
\label{sec:bg_intro}

An estimation of the background rates is needed to both select a proper threshold for
burst search, and correct the raw burst counts by subtraction of background counts.

The recorded stream of timestamps is the result of two processes: one characterized
by a high count rate, due to fluorescence photons of single molecules crossing the
excitation volume, and another characterized by a lower count rate, due to "background
counts" originating from detector dark counts, afterpulsing, out-of-focus molecules
and sample scattering and/or impurities~\cite{Edman_1996,Gopich_2008}.
The signature of these two types of processes can be
observed in the inter-photon delays distribution (i.e. the waiting times
between two subsequent timestamps) as illustrated in figure~\ref{fig:bg_dist_all}(a).
The “tail” of the distribution (a straight line in semi-log scale) corresponds
to exponentially-distributed time-delays, indicating that those counts are generated by a
Poisson process. At short
timescales, the distribution departs from the exponential due to the contribution
of the higher rate process of single molecules traversing the excitation volume.
To estimate the background rate (i.e. the inverse of the exponential time constant),
it is necessary to define a time-delay threshold above which the distribution
can be considered exponential.
Finally, a parameter estimation method needs to be specified, such as Maximum
Likelihood Estimation (MLE) or non-linear least squares curve fitting of 
the time-delay histogram (both supported in FRETBursts).
