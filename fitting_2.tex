For more examples on fitting bursts data and plotting results, refer to the
fitting section of the μs-ALEX notebook (\href{http://nbviewer.jupyter.org/github/tritemio/FRETBursts_notebooks/blob/master/notebooks/FRETBursts%20-%20us-ALEX%20smFRET%20burst%20analysis.ipynb#FRET-fit:-in-depth-example}{link}),
the \textit{Fitting Framework} section of the documentation
(\href{http://fretbursts.readthedocs.org/en/latest/fit.html}{link})
as well as the documentation for \verb|bursts_fitter| function
(\href{http://fretbursts.readthedocs.org/en/latest/plugins.html#fretbursts.burstlib_ext.bursts_fitter}{link}).

\paragraph{Python details}

Models returned by FRETBursts's factory functions (\verb|mfit.factory_*|)
are \verb|lmfit.Model| objects (\href{https://lmfit.github.io/lmfit-py/model.html}{link}).
Custom models can be created by calling \verb|lmfit.Model| directly.
When an \verb|lmfit.Model| is fitted, it returns a \verb|ModelResults| object
(\href{https://lmfit.github.io/lmfit-py/model.html#the-modelresult-class}{link}),
which contains all information related to the fit (model, data,
parameters with best values and uncertainties) and useful methods to operate on fit results.
FRETBursts puts a \verb|ModelResults| object of each excitation spot in the list
\verb|ds.E_fitter.fit_res|.
For instance, to obtain the reduced $\chi^2$ value of the E histogram fit in a
single-spot measurement \verb|d|, we use the following command:

\begin{lstlisting}
d.E_fitter.fit_res[0].redchi
\end{lstlisting}

Other useful attributes are \verb|aic| and \verb|bic| which contain
statistics for the Akaike information criterion (AIC)~\cite{akaike_new_1974} 
and the Bayes Information criterion (BIC)~\cite{schwarz_estimating_1978}.
AIC and BIC are general-purpose statistical criteria for comparing the
suitability of multiple non-nested models according to the data.
By penalizing models with higher number of parameters, these criteria 
strike a balance between the need of achieving high goodness of fit 
with the need of keeping the model complexity low to avoid overfitting.

Examples of definition and modification of fit models are provided in
the aforementioned μs-ALEX notebook
(\href{http://nbviewer.jupyter.org/github/tritemio/FRETBursts_notebooks/blob/master/notebooks/FRETBursts%20-%20us-ALEX%20smFRET%20burst%20analysis.ipynb#FRET-fit:-in-depth-example}{link}).
Users can also refer to the comprehensive lmfit's documentation
(\href{http://lmfit.github.io/lmfit-py/}{link}).

\subsection{FRET Dynamics}
\label{sec:dynamics}

Single-molecule FRET histograms show more information than just mean FRET efficiencies.
While in general the presence of several peaks clearly indicates the existence of
multiple subpopulations, a single peak cannot a priori be associated with
a single population defined by a unique FRET efficiency without further analysis.

Shot-noise analysis~\cite{Nir_2006} or probability
distribution analysis (PDA)~\cite{Antonik2006,kalinin_probability_2007}
allow to compute the minimum width of a static FRET population
(i.e. caused by the statistics of discrete photon-detection events).
Typically, several mechanisms
contribute to the broadening of the experimental FRET peak
beyond the shot-noise limit. These include heterogeneities in the sample
resulting in a distribution of Förster radii,
or actual conformational changes giving rise to a distribution
of D-A distances~\cite{sisamakis_accurate_2010}.

Gopich and Szabo developed an elegant analytical model 
for the FRET distribution of $M$ interconverting states  
based on superposition of Gaussian peaks~\cite{gopich_fret_2010}.
Unfortunately, the method is not of straightforward application for 
freely-diffusing data as it requires a special selection 
criterion for filtering bursts with quasi-Poisson rates.
Santoso et al.~\cite{santoso_probing_2009} and Kalinin et al.~\cite{Kalinin2010}
extended the PDA approach to estimate conversion rates between different
states by comparing FRET histograms as a function of the time-bin size.
In addition, Gopich and Szabo~\cite{Gopich2009, gopich_theory_2011} developed
a related method to compute conversion rates using
a likelihood function which depends on photon timestamps (overcoming
the time binning and FRET histogramming step and directly applicable 
to freely-diffusing data).
In case of measurement including lifetime, the multiparameter fluorescence
detection (MFD) method allows to identify dynamics from the deviation 
from the linear relation between lifetime and E~\cite{sisamakis_accurate_2010}.
Hoffman et al.~\cite{hoffmann_quantifying_2011} proposed a method 
called RASP (recurrence analysis of single particles) to extend 
the timescale of detectable kinetics.
Hoffman et al. compute the probability that two nearby bursts are due to
the same molecule and therefore allows setting a time-threshold
for considering consecutive bursts as the same single-molecule event.

Other interesting approaches include combining smFRET and FCS 
for detecting and quantifying kinetics on timescales much shorter 
than the diffusion 
time~\cite{laurence_correlation_2007,torres_measuring_2007,nettels_unfolded_2008}.
In addition, Bayes-based methods have been proposed to fit static
populations~\cite{devore_classic_2012,murphy_bayesian_2014}, or to study dynamics~\cite{kou_bayesian_2005}.

Finally, two related methods for discriminating between static heterogeneity
and sub-millisecond dynamics are Burst Variance Analysis
(BVA) proposed by Torella et al.~\cite{Torella_2011} and
two-channel kernel density estimator (2CDE) proposed by 
Tomov et al.~\cite{Tomov_2012}. The BVA method is described in the next section.
The 2CDE method, which has been implemented in FRETBursts, computes local
photon rates from timestamps within bursts using
Kernel Density Estimation (KDE)
(FRETBursts includes general-purpose functions
to compute KDE of photon timestamps in the \verb|phrates| module, 
(\href{http://fretbursts.readthedocs.io/en/latest/phrates.html}{link})).
From time variations of local rates, it is possible to
infer the presence of some dynamics. In particular, the 2CDE method
builds, for each burst, a quantity $(E)_D$ (or $(1-E)_A$), which is equal 
to the burst average $E$ when no dynamics is present, but is biased 
toward an higher (or lower) value in presence of dynamics. From these
quantities, a burst ``estimator'' 
(called FRET-2CDE) is derived. For a user, the 2CDE method amounts
to plotting the 2-D histogram of $E$ versus FRET-2CDE, and assessing the vertical position of the various populations: 
populations centered around FRET-2CDE=10 undergo
no dynamics, while population biased towards higher FRET-2CDE values 
undergo dynamics.

The BVA and 2CDE methods are implemented 
in two notebooks included with FRETBursts
(\href{http://nbviewer.jupyter.org/github/tritemio/FRETBursts_notebooks/blob/master/notebooks/Example%20-%20Burst%20Variance%20Analysis.ipynb}{BVA link},
\href{http://nbviewer.jupyter.org/github/tritemio/FRETBursts_notebooks/blob/master/notebooks/Example%20-%202CDE%20Method.ipynb}{2CDE link}).
To use them, a user needs to download the relevant notebook
and run the anaysis therein.
The other methods mentioned in this section are not currently 
implemented in FRETBursts.
However, users can easily implement any additional methods
in FRETBursts, using its built-in burst analysis
and timestamps/bursts manipulation functions.
In the next section,
we show how to perform low-level analysis of timestamps and bursts data 
by implementing the BVA method from scratch.
An additional example showing how to split bursts in constant time-bins
can be found in the respective FRETBursts notebook
(\href{http://nbviewer.jupyter.org/github/tritemio/FRETBursts_notebooks/blob/master/notebooks/Example%20-%20Working%20with%20timestamps%20and%20bursts.ipynb}{link}).
These examples serve as a guide for implementing new methods.
We welcome researchers willing to implement new methods to ask questions
on GitHub or on the mailing list. 
We also encourage sharing eventual new methods implemented in FRETBursts 
for the benefit the entire community.

