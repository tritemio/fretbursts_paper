For more examples on fitting bursts data and plotting results, refer to the
fitting section of the μs-ALEX notebook (\href{http://nbviewer.jupyter.org/github/tritemio/FRETBursts_notebooks/blob/master/notebooks/FRETBursts%20-%20us-ALEX%20smFRET%20burst%20analysis.ipynb#FRET-fit:-in-depth-example}{link}),
the \textit{Fitting Framework} section of the documentation
(\href{http://fretbursts.readthedocs.org/en/latest/fit.html}{link})
as well as the documentation for \verb|bursts_fitter| function
(\href{http://fretbursts.readthedocs.org/en/latest/plugins.html#fretbursts.burstlib_ext.bursts_fitter}{link}).

\paragraph{Python details}

Models returned by FRETBursts's factory functions (\verb|mfit.factory_*|)
are \verb|lmfit.Model| objects (\href{https://lmfit.github.io/lmfit-py/model.html}{link}).
Custom models can be created by calling \verb|lmfit.Model| directly.
When an \verb|lmfit.Model| is fitted, it returns a \verb|ModelResults| object
(\href{https://lmfit.github.io/lmfit-py/model.html#the-modelresult-class}{link}),
which contains all information related to the fit (model, data,
parameters with best values and uncertainties) and useful methods to operate on fit results.
FRETBursts puts a \verb|ModelResults| object of each excitation spot in the list
\verb|ds.E_fitter.fit_res|.
For instance, to obtain the reduced $\chi^2$ value of the E histogram fit in a
single-spot measurement \verb|d|, we use the following command:

\begin{lstlisting}
d.E_fitter.fit_res[0].redchi
\end{lstlisting}

Other useful attributes are \verb|aic| and \verb|bic| which contain
respectively the Akaike information criterion (AIC)~\cite{akaike_new_1974} 
and the Bayes Information criterion (BIC)~\cite{schwarz_estimating_1978}.
AIC and BIC are general-purpose statistical criteria for comparing 
models and selecting the most appropriate for a given dataset.
By penalizing models with higher number of parameters, these criteria 
strike a balance between the need of achieving high goodness of fit 
with the need of keeping the model complexity low to avoid overfitting.

Examples of definition and modification of fit models are provided in
the aforementioned μs-ALEX notebook
(\href{http://nbviewer.jupyter.org/github/tritemio/FRETBursts_notebooks/blob/master/notebooks/FRETBursts%20-%20us-ALEX%20smFRET%20burst%20analysis.ipynb#FRET-fit:-in-depth-example}{link}).
Users can also refer to the comprehensive lmfit's documentation
(\href{http://lmfit.github.io/lmfit-py/}{link}).

\subsection{FRET Dynamics}
\label{sec:dynamics}

FRET peaks resolved from FRET histograms correspond to different FRET
populations. However determining whether these populations are due to static
heterogeneity (different D-A distances, Förster radius, etc.) or a
dynamic equilibrium, requires further analysis.

Shot-noise analysis (SNA)~\cite{Nir_2006} or probability
distribution analysis (PDA)~\cite{Antonik2006} allow to compute
the minimal width of a FRET peak originating from a single
FRET efficiency. Typically, several mechanisms
contribute to the broadening of the experimental FRET peak
beyond the shot-noise limit. For example, heterogeneities in the sample
(resulting in a distribution of Förster radiuses)
or actual conformational changes (resulting in a distribution
of D-A distances) can cause FRET peak broadening~\cite{sisamakis_accurate_2010}.

Kalinin~\cite{Kalinin2010} and Santoso~\cite{santoso_probing_2009}
extended the PDA approach to estimate conversion rates between different
states by comparing FRET histograms as a function of the time-bin size.
Gopich and Szabo~\cite{Gopich2009, gopich_theory_2011} developed
a related method to compute conversion rates using
a likelihood function which depends on photon timestamps (overcoming
the time binning and FRET histogramming step).
Hoffman~\cite{hoffmann_quantifying_2011} proposed a method 
called RASP (recurrence analysis of single particles) to extend 
the timescale of detectable kinetics.
Hoffman computes the probability that two nearby bursts are originating from
the same molecule and therefore allows to set a time-threshold
for considering consecutive bursts as originating from the same molecule.

Other two related methods for discriminating between static heterogeneity
and sub-millisecond dynamics are Burst Variance Analysis
(BVA) proposed by Torella~\cite{Torella_2011} and
kernel density distribution estimator (2CDE) proposed by 
Tomov~\cite{Tomov_2012}. The latter method computes local
photon rates from timestamps within bursts using
Kernel Density Estimation (KDE)
(FRETBursts includes general-purpose functions
to compute KDE of photon timestamps in the \verb|phrates| module, 
(\href{http://fretbursts.readthedocs.io/en/latest/phrates.html}{link})).

The BVA and 2CDE methods are implemented 
in two notebooks included with FRETBursts
(\href{http://nbviewer.jupyter.org/github/tritemio/FRETBursts_notebooks/blob/master/notebooks/Example%20-%20Burst%20Variance%20Analysis.ipynb}{BVA link},
\href{http://nbviewer.jupyter.org/github/tritemio/FRETBursts_notebooks/blob/master/notebooks/Example%20-%202CDE%20Method.ipynb}{2CDE link}).
To use them, a user needs to download the relevant notebook
and run the anaysis therein.
The other methods mentioned in this section are not currently 
implemented in FRETBursts.
However, users can implement their additional favorite method
taking advantage of FRETBursts functions for burst analysis
and timestamps/bursts manipulation.
To facilitate this task, in the next section,
we show how to perform low-level analysis of timestamps and bursts data 
by implementing the BVA method from scratch.
An additional example showing how to split bursts in constant time-bins
can be found in the respective FRETBursts notebook
(\href{http://nbviewer.jupyter.org/github/tritemio/FRETBursts_notebooks/blob/master/notebooks/Example%20-%20Working%20with%20timestamps%20and%20bursts.ipynb}{link}).
These examples serve as a guide for implementing new methods.
We welcome researchers willing to implement new methods to ask questions
on GitHub or on the mailing list. 
We also encourage sharing eventual new methods implemented in FRETBursts 
for the benefit the entire community.

