\section{Introduction}

\subsection{Open Science and Reproducibility}

In the last 20 years, single molecule FRET (smFRET), has emerged as one of the most
useful techniques in single-molecule spectroscopy~\cite{Weiss_1999,Hohlbein_2014}.
Except for a few advanced time-resolved techniques~\cite{Lerner_2014,Rahamim_2015},
smFRET unique ability is resolving conformational
changes of biomolecules or measuring binding-unbinding kinetics on heterogeneous samples.
smFRET measurements on freely diffusing molecules (the focus of this paper) have the advantage
of probing molecules and processes without perturbation from surface
immobilization~\cite{Dahan_1999,Eggeling_1998}.

The field of freely-diffusing smFRET data analysis, has seen a number of significant
contributions over the
years~\cite{Fries_1998,Eggeling_2001,Zhang_2005,Gopich_2005,Lee_2005,Nir_2006,Antonik2006,Gopich_2007,Gopich_2008,Camley_2009,Santoso_2010,Torella_2011,Tomov_2012}.
Historically, each research group have implemented its own private version
of analysis software (oftentimes highly dependent on a particular setup
configuration) with almost no collaboration or code sharing.
Even in our group, past smFRET papers merely mention the usage of custom-made software without
additional details~\cite{Lee_2005,Nir_2006}.
This situation makes it hard to reproduce and validate and to build upon
results from different groups.
Moreover, as new methods are proposed in literature,
it is oftentimes hard to quantify their performances.
An independent quantitative assessment
would require a complete reimplementation, an effort a few groups can afford.
As a results, potentially useful analysis improvements
are rarely adopted by the community at large.
In contrast with other consolidated traditions such as
sharing protocols and samples, for scientific software,
we have relegated ourselves to islands of non-communication.

From a more general stance, non-availability of codes
used for scientific results, hinders reproducibility,
makes it impossible to review and validate the software correctness
and prevents improvements and extensions by other scientists.
This situation, common in many disciplines,
represents a real impediment to the scientific progress.
Since pioneering work of Donoho group in the 90's~\cite{Buckheit_1995},
it has become evident that developing and maintaining open source scientific software
for reproducible research is a critical requirement of modern
scientific enterprice~\cite{Ince_2012,Vihinen_2015}.

%Peer-reviewed publications describing such software are also necessary~\cite{Pradal_2013},
%although the debate is still open on the most effective model for peer-reviewing this
%class of publications~\cite{Check_Hayden_2013,Check_Hayden_2015}
%(\href{https://software-carpentry.org/blog/2015/04/quality-is-free-getting-there-isnt.html}{Willson 2015})
%(\href{https://www.mozillascience.org/effective-code-review-for-journals}{Mills 2015})
%(\href{http://ivory.idyll.org/blog/2015-we-live-in-a-bubble.html}{Brown 2015} and \href{http://ivory.idyll.org/blog/on-code-review-of-scientific-code.html}{2013}).

Facing these issues, we developed FRETBursts,
an open source Python software for burst analysis of freely-diffusing
single-molecule FRET experiments.
With FRETBursts we provide a tool that is available to any scientist
to use, inspect and modify. FRETBursts is suitable for routine state-of-the-art
analysis of smFRET data but also represents an ideal platform
for quantitative comparison of different methods in burst analysis.
To facilitate reproducibility of complete analysis
workflows, FRETBursts execution model is based on Juyter Notebook~\cite{Shen_2014}.
A notebook contains a narrative, input parameters, code and 
results in a single document that is easy to share and re-execute.
To minimize chance of bugs we employ modern software engineering techniques
such as unit testing and continuous integration.
FRETBursts is hosted and openly developed on GitHub~\cite{Blischak_2016,Prli__2012},
where users can send comments, report issues or contribute code.
In a parallel effort, we recently introduced Photon-HDF5,
a open file format for timestamp-based single-molecule fluorescence
experiments~\cite{Ingargiola2016}. Together with Photon-HDF5,
FRETBursts contributes to the ecosystem
of open tools for reproducible science in the single-molecule field.

\subsection{Paper Overview}
This paper is an introduction to smFRET burst analysis and FRETBursts usage.
Therefore, after a brief overview of FRETBursts features (section~\ref{sec:overview}),
we introduce core smFRET burst analysis concepts and terminology
(section~\ref{sec:concepts}). These concepts are used throughout the paper
so reading section~\ref{sec:concepts} is highly recommended.

In section~\ref{sec:analysis}, we illustrate the practical steps involved
in smFRET burst analysis: data loading (section~\ref{sec:dataload}), defining
excitation alternation periods (section~\ref{sec:alternation}), background
correction (section~\ref{sec:bg_calc}), burst search (section~\ref{sec:burstsearch}),
burst selection (section~\ref{sec:burstsel}) and FRET fitting (section~\ref{sec:fretfit}).
The aim is elucidating the specificities and trade-off of various approaches
with enough details to empowers reader new to the field to customize the analysis to their own needs.
For the most advanced use-case, section~\ref{sec:bva} walks the reader thorough implementing
Burst Variance Analysis (BVA)~\cite{Torella_2011} as an example of manipulating timestamps
and burst data.
Finally, in section~\ref{sec:conclusions}, we summarize what we believe to be
the strengths of FRETBursts software.

Throughout this paper,
links to relevant sections of documentation and other web resources
are displayed as ``(link)''.
In order to make the text accessible to the widest number of readers,
we concentrated python-specific details in special subsections titled
\textit{Python details}. These subsections provide deeper insights for readers
already familiar with python and can be safely skipped otherwise.
Finally, note that all commands here reported can be found in the
accompanying notebooks
(\href{https://github.com/tritemio/fretbursts_paper}{link}).
