\section{Introduction}

\subsection{Open Science and Reproducibility}

Over the past 20 years, single molecule FRET (smFRET) has grown into one of the most
useful techniques in single-molecule spectroscopy~\cite{Weiss_1999,Hohlbein_2014}.
While it is possible to extract information on sub-populations using ensemble measurements 
(e.g. ~\cite{Lerner_2014,Rahamim_2015}),
smFRET unique feature is its ability to very straightforwardly resolve conformational
changes of biomolecules or measure binding-unbinding kinetics in heterogeneous 
samples~\cite{Selvin_2000,Roy_2008,Schuler_2008,Sisamakis_2010,Haran_2012}.
smFRET measurements on freely diffusing molecules (the focus of this paper) 
have the additional advantage over measurements performed on immobilized molecules, 
of allowing to probe molecules and processes without perturbation from surface
immobilization or additional functionalization needed for surface 
attachment~\cite{Eggeling_1998,Dahan_1999}.

The increasing amount of work using freely-diffusing smFRET has motivated 
a growing number of theoretical contributions to the specific topic of data 
analysis~\cite{Fries_1998,Eggeling_2001,Zhang_2005,Gopich_2005,Lee_2005,Nir_2006,Antonik2006,Gopich_2007,Gopich_2008,Camley_2009,Santoso_2010,Torella_2011,Tomov_2012}. 
Despite this profusion of publications, most research groups still rely on 
their own implementation of a limited number of methods, with very little 
collaboration or code sharing. 
To clarify this statement, let us point that our own group's past smFRET papers 
merely mention the use of custom-made software without additional details~\cite{Lee_2005,Nir_2006}. 
Even though some of these software tools are made available upon request, 
or sometimes shared publicly on websites, 
it remains hard to reproduce and validate results from different groups, 
let alone build upon them.
Additionally, as new methods are proposed in literature,
it is generally difficult to quantify their performance compared to other methods.
An independent quantitative assessment
would require a complete reimplementation, an effort few groups can afford.
As a result, potentially useful analysis improvements
are either rarely or slowly adopted by the community.
In contrast with other established traditions such as
sharing protocols and samples, in the domain of scientific software,
we have relegated ourselves to islands of non-communication.

From a more general standpoint, the non-availability of the code
used to produce scientific results, hinders reproducibility,
makes it impossible to review and validate the software's correctness
and prevents improvements and extensions by other scientists.
This situation, common in many disciplines,
represents a real impediment to the scientific progress.
Since pioneering work of Donoho group in the 90's~\cite{Buckheit_1995},
it has become evident that developing and maintaining open source scientific software
for reproducible research is a critical requirement of the modern
scientific enterprise~\cite{Ince_2012,Vihinen_2015}.

%Peer-reviewed publications describing such software are also necessary~\cite{Pradal_2013},
%although the debate is still open on the most effective model for peer-reviewing this
%class of publications~\cite{Check_Hayden_2013,Check_Hayden_2015}
%(\href{https://software-carpentry.org/blog/2015/04/quality-is-free-getting-there-isnt.html}{Willson 2015})
%(\href{https://www.mozillascience.org/effective-code-review-for-journals}{Mills 2015})
%(\href{http://ivory.idyll.org/blog/2015-we-live-in-a-bubble.html}{Brown 2015} and \href{http://ivory.idyll.org/blog/on-code-review-of-scientific-code.html}{2013}).

Other disciplines have started tackling this issue 
(cite http://biorxiv.org/content/early/2016/03/24/045104 once doi activated),
and even in the single-molecule field a few recent publications have included or presented 
open source software  
for analysis of surface immobilized experiments~\cite{Bronson_2009,Greenfeld_2012, K_nig_2013}.
For freely-diffusing smFRET experiments, although it is common to find mention of 
"code available from the authors upon request" in publications, there is a dearth 
of such open source code, with, to our knowledge, the notable exception of a single 
example~\cite{Murphy2014}.
To address this issue, we have developed FRETBursts,
an open source Python software for analysis of freely diffusing single-molecule FRET measurement.
FRETBursts can be used, inspected and modified by anyone interested in using 
state-of-the art smFRET analysis methods or implementing modifications or completely new techniques. 
FRETBursts therefore represents an ideal platform
for quantitative comparison of different methods for smFRET burst analysis.
Technically, a strong emphasis has been given to the reproducibility of complete analysis
workflows. FRETBursts uses Jupyter Notebooks~\cite{Shen_2014},
an interactive and executable document containing textual narrative, input parameters, 
code, and computational results (tables, plots, etc.). A notebook thus captures the various analysis steps
in a document which is easy to share and execute.
To minimize the possibility of bugs being introduced inadvertently~\cite{Soergel_2015} 
we employ modern software engineering techniques
such as unit testing and continuous integration~\cite{Wilson_2014}
(cite http://biorxiv.org/content/early/2016/03/24/045104 once doi activated).
FRETBursts is hosted on GitHub~\cite{Blischak_2016,Prli__2012},
where users can write comments, report issues or contribute code.
In a related effort, we recently introduced Photon-HDF5~\cite{Ingargiola2016},
a open file format for timestamp-based single-molecule fluorescence
experiments. An other related open source tool is PyBroMo~\cite{Ingargiola_2016},
a freely-diffusing smFRET simulator which produces Photon-HDF5 files that are
directly analyzable with FRETBursts.
Together with all the aforementioned tools, FRETBursts contributes to the growing 
ecosystem of open tools for reproducible science in the single-molecule field.

\subsection{Paper Overview}
This paper is written as an introduction to smFRET burst analysis and its implementation in FRETBursts.
After a brief overview of FRETBursts features (section~\ref{sec:overview}),
we introduce essential concepts and terminology for smFRET burst analysis 
(section~\ref{sec:concepts}).

In section~\ref{sec:analysis}, we illustrate the steps involved
in smFRET burst analysis: (i) data loading (section~\ref{sec:dataload}), (ii) definition of the
excitation alternation periods (section~\ref{sec:alternation}), (iii) background
correction (section~\ref{sec:bg_calc}), (iv) burst search (section~\ref{sec:burstsearch}),
(v) burst selection (section~\ref{sec:burstsel}) and (vi) FRET histogram fitting (section~\ref{sec:fretfit}).
The aim of this section is to illustrate the specificities and trade-off involved in various approaches
with sufficient details to enable readers new to the field (or Jupyter Notebooks) 
to customize the analysis for their own needs.
Section~\ref{sec:bva} walks the reader thorough implementing
Burst Variance Analysis (BVA)~\cite{Torella_2011}, as an example of implementation 
of an advanced data processing technique.
Finally, section~\ref{sec:conclusions} summarizes what we believe to be
the strengths of FRETBursts software.

Throughout this paper,
links to relevant sections of documentation and other web resources
are displayed as ``(link)''.
In order to make the text more legible,
we have concentrated Python-specific details in subsections entitled
\textit{Python details}. These subsections provide deeper insights for readers
already familiar with Python and can be initially skipped by readers who are not.
Finally, note that all commands here reported can be found in the
accompanying notebooks
(\href{https://github.com/tritemio/fretbursts_paper}{link}).
