\section{Introduction}

\subsection{Open Science and Reproducibility}

Over the past 20 years, single molecule FRET (smFRET) has grown into one of the most
useful techniques in single-molecule spectroscopy~\cite{Weiss_1999,Hohlbein_2014}.
While it is possible to extract information on sub-populations using ensemble measurements (e.g. ~\cite{Lerner_2014,Rahamim_2015}),
smFRET unique feature is its ability to very straightforwardly resolve conformational
changes of biomolecules or measure binding-unbinding kinetics in heterogeneous samples.
smFRET measurements on freely diffusing molecules (the focus of this paper) have the additional advantage over measurements performed on immobilized molecules, of allowing to probe molecules and processes without perturbation from surface
immobilization or additional functionalization needed for surface attachment~\cite{Dahan_1999,Eggeling_1998}.

The increasing amount of work using freely-diffusing smFRET has resulted in a growing number of theoretical contributions to the specific topic of data analysis~\cite{Fries_1998,Eggeling_2001,Zhang_2005,Gopich_2005,Lee_2005,Nir_2006,Antonik2006,Gopich_2007,Gopich_2008,Camley_2009,Santoso_2010,Torella_2011,Tomov_2012}. Despite this profusion of methods, most research groups still rely on their own implementation of a limited number of methods, with very little collaboration or code sharing. To clarify this statement, let us point that our own group's past smFRET papers merely mention the use of custom-made software without
additional details~\cite{Lee_2005,Nir_2006}. Even though some of these software tools are made available upon request, or shared publicly on websites (whether as source code or executable), it remains hard to reproduce and validate results from different groups, let alone build upon them. 
Additionally, as new methods are proposed in literature,
it is generally difficult to quantify their performance compared to other methods.
An independent quantitative assessment
would require a complete reimplementation, an effort few groups can afford.
As a result, potentially useful analysis improvements
are either rarely or slowly adopted by the community.
In contrast with other established traditions such as
sharing protocols and samples, in the domain of scientific software,
we have relegated ourselves to islands of non-communication.

From a more general stance, non-availability of codes
used for scientific results, hinders reproducibility,
makes it impossible to review and validate the software correctness
and prevents improvements and extensions by other scientists.
This situation, common in many disciplines,
represents a real impediment to the scientific progress.
Since pioneering work of Donoho group in the 90's~\cite{Buckheit_1995},
it has become evident that developing and maintaining open source scientific software
for reproducible research is a critical requirement of the modern
scientific enterprise~\cite{Ince_2012,Vihinen_2015}.

%Peer-reviewed publications describing such software are also necessary~\cite{Pradal_2013},
%although the debate is still open on the most effective model for peer-reviewing this
%class of publications~\cite{Check_Hayden_2013,Check_Hayden_2015}
%(\href{https://software-carpentry.org/blog/2015/04/quality-is-free-getting-there-isnt.html}{Willson 2015})
%(\href{https://www.mozillascience.org/effective-code-review-for-journals}{Mills 2015})
%(\href{http://ivory.idyll.org/blog/2015-we-live-in-a-bubble.html}{Brown 2015} and \href{http://ivory.idyll.org/blog/on-code-review-of-scientific-code.html}{2013}).

Facing these issues, we developed FRETBursts,
an open source Python software for burst analysis of freely-diffusing
single-molecule FRET experiments.
With FRETBursts we provide a tool that is available to any scientist
to use, inspect and modify. FRETBursts is suitable for routine state-of-the-art
analysis of smFRET data but also represents an ideal platform
for quantitative comparison of different methods in burst analysis.
To facilitate reproducibility of complete analysis
workflows, FRETBursts execution model is based on Jupyter Notebook~\cite{Shen_2014}.
A notebook contains a narrative, input parameters, code and 
results in a single document that is easy to share and re-execute.
To minimize the inevitable occurrence of bugs~\cite{Soergel_2015} 
we employ modern software engineering techniques
such as unit testing and continuous integration.
FRETBursts is hosted and openly developed on GitHub~\cite{Blischak_2016,Prli__2012},
where users can send comments, report issues or contribute code.
In a parallel effort, we recently introduced Photon-HDF5,
a open file format for timestamp-based single-molecule fluorescence
experiments~\cite{Ingargiola2016}. Together with Photon-HDF5,
FRETBursts contributes to the ecosystem
of open tools for reproducible science in the single-molecule field.

\subsection{Paper Overview}
This paper is an introduction to smFRET burst analysis and FRETBursts usage.
Therefore, after a brief overview of FRETBursts features (section~\ref{sec:overview}),
we introduce core smFRET burst analysis concepts and terminology
(section~\ref{sec:concepts}). These concepts are used throughout the paper
so reading section~\ref{sec:concepts} is highly recommended.

In section~\ref{sec:analysis}, we illustrate the practical steps involved
in smFRET burst analysis: data loading (section~\ref{sec:dataload}), defining
excitation alternation periods (section~\ref{sec:alternation}), background
correction (section~\ref{sec:bg_calc}), burst search (section~\ref{sec:burstsearch}),
burst selection (section~\ref{sec:burstsel}) and FRET fitting (section~\ref{sec:fretfit}).
The aim is elucidating the specificities and trade-off of various approaches
with enough details to empower readers new to the field to customize the analysis to their own needs.
For the most advanced use-case, section~\ref{sec:bva} walks the reader thorough implementing
Burst Variance Analysis (BVA)~\cite{Torella_2011} as an example of manipulating timestamps
and burst data.
Finally, in section~\ref{sec:conclusions}, we summarize what we believe to be
the strengths of FRETBursts software.

Throughout this paper,
links to relevant sections of documentation and other web resources
are displayed as ``(link)''.
In order to make the text accessible to the widest number of readers,
we concentrated python-specific details in special subsections titled
\textit{Python details}. These subsections provide deeper insights for readers
already familiar with python and can be safely skipped otherwise.
Finally, note that all commands here reported can be found in the
accompanying notebooks
(\href{https://github.com/tritemio/fretbursts_paper}{link}).
