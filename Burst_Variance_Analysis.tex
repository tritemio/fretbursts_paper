\section{Implementing Burst Variance Analysis}

\label{sec:bva}
In this section, we describe how to implement burst variance analysis (BVA)
as described in~\cite{Torella_2011}.
FRETBursts provides well-tested, general-purpose functions for timestamps and burst data
manipulation and therefore simplifies implementing custom burst analysis algorithms such as BVA.

\subsection{BVA Overview}
The BVA method has been developed to identify the presence of dynamics
in FRET distributions~\cite{Torella_2011},
and has been successfully applied to identify biomolecular processes with
dynamics on the millisecond time-scale~\cite{Torella_2011, Robb_2013}.

The basic idea behind BVA is to subdivide bursts into contiguous burst chunks (sub-bursts)
comprising a fixed number $n$ of photons,
and to compare the empirical variance of acceptor counts of all sub-bursts in a burst,
with the theoretical shot-noise-limited variance.
An empirical variance of sub-bursts larger than the shot-noise-limited value indicates
the presence of dynamics.

In a FRET (sub-)population originating from a single static FRET efficiency,
the sub-bursts acceptor counts $n_a$ can be modeled as a binomial-distributed random variable 
$N_a \sim \operatorname{B}(n, E_p)$, where $n$ is the number of photons in each sub-burst and 
$E_p$ is the estimated population proximity-ratio (PR). 
Note that we can use the PR because, regardless of the molecular FRET efficiency, 
the detected counts are partitioned between donor and acceptor channels according to
a binomial distribution with success probability equal to the PR.
The only approximation done here is neglecting the presence of background
(a reasonable approximation since the backgrounds counts are in general a 
very small fraction of the total counts). 
We refer the interested reader to~\cite{Torella_2011} for further discussion.

If $N_a$ follows a binomial distribution, the random variable $E_{\textrm{sub}} = N_a/n$,
has a standard deviation reported in eq.~\ref{eq:binom_std}. 

\begin{equation}
\label{eq:binom_std}
\operatorname{Std}(E_{\textrm{sub}}) = \left( \frac{E_p\,(1 - E_p)}{n} \right)^{1/2}
\end{equation}

BVA analysis consists in four steps: 1) dividing bursts into consecutive sub-bursts 
containing a constant number of consecutive photons~\textit{n}, 2) computing the PR 
of each sub-burst, 3) calculating the empirical standard deviation ($s_E$) of sub-bursts
PR in each burst, and 4) comparing $s_E$ to the expected standard deviation 
of a shot-noise-limited distribution~(eq.~\ref{eq:binom_std}).
If, as in figure~\ref{fig:bva_static}, the observed FRET efficiency distribution 
originates from a static mixture of sub-populations (of different 
non-interconverting molecules) characterized by distinct FRET efficiencies, 
$s_E$ of each burst is only affected by shot-noise and will follow the expected 
standard deviation curve based on eq.~\ref{eq:binom_std}. 
Conversely, if the observed distribution originates from biomolecules belonging to a single species, 
which interconverts between different FRET sub-populations (over times comparable to the diffusion 
time), as in figure~\ref{fig:bva_dynamic}, $s_E$ of each burst will be larger than the expected 
shot-noise-limited standard deviation, and will be located above the shot-noise standard 
deviation curve (right panel of figure~\ref{fig:bva_dynamic}).
