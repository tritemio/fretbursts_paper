\section{Implementing Burst Variance Analysis}

\label{sec:bva}
In this section, we describe how to implement burst variance analysis (BVA) as described in ~\cite{Torella_2011},
as an example of how to extend the capabilities of FRETBursts.
%FRETBursts provides well-tested, general-purpose functions for timestamps and burst data 
%manipulation and therefore simplifies implementing custom burst analysis algorithms such as BVA.

\subsection{BVA Overview}
Single-molecule FRET histograms show more information than just mean FRET efficiencies. 
While in general the presence of several peaks clearly indicates the existence of multiple subpopulations, 
a single peak cannot a priori be associated with a single population defined by a unique FRET efficiency without further analysis (such as, for instance, shot-noise analysis ~\cite{Nir_2006,Antonik2006}).

The FRET histogram of a single FRET population has a minimum width set by shot noise (i.e. the width is caused by
the statistics of discrete photon-detection events). FRET distributions broader than the shot noise limit, 
can be ascribed to either a static mixture of species with slightly different FRET efficiencies, 
or to a specie undergoing dynamic transitions (e.g. interconversion between multiple states,
diffusion in a continuum of conformations, binding-unbinding events, etc...).
When the single peak of a FRET distribution is wider than predicted from shot-noise, 
it is not possible to discriminate between the static and dynamic case without further analysis.
The BVA method has been developed to address this issue, namely identifying the presence of dynamics 
in FRET distributions~\cite{Torella_2011}, 
and has been successfully applied to identify biomolecular processes with 
dynamics on the millisecond time-scale~\cite{Torella_2011, Robb_2013}.

The basic idea behind BVA is to subdivide bursts into contiguous burst chunks comprising a fixed number $n$ of photons,
and to compare the empirical variance of acceptor counts across all sub-bursts in a burst 
with the theoretical shot-noise limited variance, as expected from a binomial distribution.
An empirical variance of burst chunks larger than the shot-noise limited value indicates
the presence of dynamics. Since the estimation of the burst chunks variance is affected
by uncertainty, BVA analysis provides and indication of an higher or lower probability
of observing dynamics.

In a FRET (sub-)population originating from a single static FRET efficiency,
the burst chunks acceptor counts $N_a$ can be modeled as a binomial-distributed random variable 
$N_a \sim \operatorname{Binom} \{n, E\}$, where $n$ is the number of photons in each burst chunk and 
$E$ is the estimated population FRET efficiency. Note that, without approximation, we can replace 
E with PR and use the uncorrected counts. This is possible because, regardless of the 
molecular FRET efficiency, the detected counts are partitioned between donor and acceptor channel
according to a binomial distribution with a $p$ parameter equal to PR.
The only approximation done here is neglecting the presence background
(a reasonable approximation since the backgrounds counts are in general a very small fraction of
the total counts). 
We refer the interested reader to~\cite{Torella_2011} for further discussion.

If $N_a$ follows a binomial distribution, the random variable $E = N_a/n$,
has a standard deviation reported in eq.~\ref{eq:binom_std}. 

\begin{equation}
\label{eq:binom_std}
\operatorname{Std(\textit{E})} = {\sqrt{\frac{E(1 - E)}{n}}}
\end{equation}

BVA analysis consists of four steps: 1) dividing bursts into consecutive burst chunks containing a constant number of consecutive photons,~\textit{n}, 2) computing the FRET efficiencies of each burst chunk, 3) calculating the empirical standard deviation ($s_E$) of burst chunks FRET efficiencies over the whole burst, and 4) comparing $s_E$ to the expected standard deviation of a shot-noise limited distribution~(eq.~\ref{eq:binom_std}).
If, as in figure~\ref{fig:bva_static}, the observed FRET efficiency distribution 
originates from a static mixture of sub-populations (of different 
non-interconverting molecules) characterized by distinct FRET efficiencies, 
$s_E$ of each burst is only affected by shot noise and will follow the expected standard deviation curve based on eq.~\ref{eq:binom_std}. 
Conversely, if the observed distribution originates from biomolecules belonging to a single specie, which 
interconverts between different FRET sub-populations (over times comparable to the diffusion 
time), as in figure~\ref{fig:bva_dynamic}, $s_E$ of each burst will be larger than the expected 
shot-noise-limited standard deviation, and will be located above the shot-noise standard 
deviation curve (right panel of figure~\ref{fig:bva_dynamic}).

\subsection{BVA Implementation}

The following paragraphs describe the low-level details involved in implementing the BVA using FRETBursts.
The main goal is to illustrate a real-world example of accessing and manipulating timestamps and burst data.
For a ready-to-use BVA implementation users can refer to the corresponding notebook included with FRETBursts
(\href{http://nbviewer.jupyter.org/github/tritemio/FRETBursts_notebooks/blob/master/notebooks/Example%20-%20Burst%20Variance%20Analysis.ipynb}{link}).

\paragraph{Python details}
For BVA implementation, two photon streams are needed: all-photons during donor excitation (Dex) 
and acceptor photons during donor excitation (DexAem). 
These photon stream selections are obtained by computing boolean masks as follows 
(see section~\ref{sec:burststimes}):

\begin{lstlisting}
Dex_mask = ds.get_ph_mask(ph_sel=Ph_sel(Dex='DAem'))   
DexAem_mask = ds.get_ph_mask(ph_sel=Ph_sel(Dex='Aem'))
DexAem_mask_d = AemDex_mask[Dex_mask]
\end{lstlisting}

Here, the first two variables (\verb|Dex_mask| and \verb|DexAem_mask|) are used to
select photon from the all-photons timestamps array,
while \verb|DexAem_mask_d|, selects A-emitted photons from the
array of photons emitted during D-excitation. As shown below, 
the latter is needed to count acceptor photons in burst chunks.

Next, the burst data relative to the D-excitation photon stream is needed (by default 
burst start-stop index refer to all-photons timestamps array):

\begin{lstlisting}
ph_d = ds_FRET.get_ph_times(ph_sel=Ph_sel(Dex='DAem'))
bursts = ds_FRET.mburst[0] 
bursts_d = bursts.recompute_index_reduce(ph_d)
\end{lstlisting}

Here, \verb|ph_d| contains the Dex timestamps, \verb|bursts| the original burst data and 
\verb|bursts_d| the burst data with start-stop indexes relative to \verb|ph_d|.

Finally, with the previous variables at hand, the BVA algorithm
can be easily implrement by computing the $s_E$ quantity for each burst:

\begin{lstlisting}
n = 7
E_sub_std = []
for burst in bursts_d:
    E_sub = []
    startlist = range(burst.istart, burst.istop + 2 - n, n)
    stoplist = [i + n for i in startlist]
    for start, stop in zip(startlist, stoplist): 
        A_D = DexAem_mask_d[start:stop].sum()
        E = A_D / n
        E_sub.append(E)
    E_sub_std.append(np.std(E_sub))
\end{lstlisting}

Here, \verb|n| is the BVA parameter defining the number of photons in each burst chunk. 
The outer loop iterates through bursts, while the inner loop iterates through burst chunks.
The variables \verb|startlist| and \verb|stoplist| are the list of start-stop indexes for
all burst chunks in current burst.
In the inner loop, \verb|A_D| and \verb|E| contain the number of acceptor photons and 
FRET efficiency for the current burst chunk. Finally, for each burst, the standard deviation 
of \verb|E| is appended to \verb|E_sub_std|.

By plotting the 2D distribution of $s_E$ (i.e. \verb|E_sub_std|) versus the average (uncorrected) E 
we obtain the BVA plots of figure~\ref{fig:bva_static} and~\ref{fig:bva_dynamic}.

