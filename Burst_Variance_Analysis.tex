\section{Implementing Burst Variance Analysis}
In this section we describe how to implement the Burst Variance Analysis (BVA)~\cite{Torella_2011},
as an example of building custom burst analysis algorithms in FRETBursts.

Single-molecule FRET histograms show more information than just mean FRET efficiencies. 
While, in general, several peaks indicate the presence of multiple subpopulations, 
a single peak cannot be associated a priori with a single FRET efficiency,
unless a detailed shot-noise analysis is carried out~\cite{Nir_2006} ADD SEIDEL PDA PAPER.
A broad FRET distribution might be attributed to a mixture of multiple species with static but different FRET efficiencies, single species with dynamic fluctuations between multiple FRET states, or a combination of the two cases. Burst Variance Analysis (BVA) is an analysis method for single molecule FRET experiments, developed to detect molecular dynamics~\cite{Torella_2011}. It has been successfully implemented to identify heterogeneities in FRET histograms due to dynamic processes of biomolecules in millisecond time scale~\cite{Torella_2011, Robb_2013}.

BVA analysis consists of four steps: 1) slicing bursts into sub-bursts containing \textit{n} consecutive photons, 2) computing FRET efficiencies of each sub-burst, 3) calculating the empirical standard deviation ($s_E$) of sub-burst FRET efficiencies over the whole burst, and 4) comparing $s_E$ to an expected standard deviation based on shot noise limited distribution~\cite{Torella_2011}. 

If the observed broadness originates from different molecules having distinct FRET efficiencies without dynamics, $s_E$ of each burst is only affected by shot noise and will follow the expected standard deviation curve rationalized by a binomial distribution (see equation 4 in~\cite{Torella_2011}). However, if the observed broadness is due to millisecond dynamics of single species of biomolecules, $s_E$ of each burst is supposed to be larger than the expected standard deviation and sit above the expected standard deviation curve as shown in figure .
Since FRETBursts is based on open source python packages, BVA can be easily built and implemented by FRETBursts with combination of other python packages (see notebook).  



see section~\ref{sec:ph_streams}