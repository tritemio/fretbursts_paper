\section{smFRET data analysis}

\subsection{Introduction}

In the following we provide a high-level description of FRETbursts usage. In general, we suggest to import FRETbursts with the expression:

\verb|import fretbursts as fb|

that will make available all the FRETBursts functions with a concise `fb.` prefix. In this article, however, we assume that FRETBursts is imported with the shortcut form:

\verb|from fretbursts import *|

that allows to skip the \verb|fb.| prefix and also imports some common numeric libraries (numpy, pandas, matplotlib.pyplot imported as \verb|np|, \verb|pd| and \verb|plt| respectively).

\subsection{Photon streams}

The fundamental data type in smFRET experiments is the array of photon timestamps. Normally, for single-spot measurements, all the timestamps are stored in a single array. In multi-spot measurements we have as many timestamps array as the number of excitation spots.

Each timestamps array contains timestamps from both the donor and the acceptor channel. Furthermore, in ALEX measurement, we differentiate between photons emitted during donor-excitation and acceptor excitation. We call different selection of photons/timestamps "photon streams". In non-ALEX smFRET data we have 3 photon streams, while in ALEX data we have 5 photon streams.