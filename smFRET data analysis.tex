\section{Architecture and concepts}

In this section we introduce some general concepts and naming conventions in FRETBursts.

\subsection{Photon streams}
\label{sec:ph_streams}

The fundamental data at the core of smFRET experiments is the array of photon timestamps. In single-spot measurements, all the timestamps are stored in a single array. In multi-spot measurements we have as many timestamps arrays as the number of excitation spots.

Each timestamps array contains timestamps from both the donor (D) and the acceptor channel (A). In ALEX measurements, we can further differentiate between photons emitted during D and A excitation periods. In FRETBursts the different selections of photons/timestamps are called "photon streams" and they are specified with a \href{http://fretbursts.readthedocs.org/en/latest/ph_sel.html}{\texttt{Ph\_sel} object} . In non-ALEX smFRET data we have 3 base photon streams (table~\ref{tab:ph_sel_smfret}), while in ALEX data we have 5 base photon streams 
(table~\ref{tab:ph_sel_alex}).

The \href{http://fretbursts.readthedocs.org/en/latest/ph_sel.html}{\texttt{Ph\_sel} class} allows to express any combination of photon streams. 
For example, in ALEX measurements, the D-emission during A-excitation stream is usually excluded because it does not contain any useful signal~\cite{Lee_2005}. To indicate all but the photons in this photon stream we write \verb|Ph_sel(Dex='DAem', Aex='Aem')|, that litteraly means \textit{select donor and acceptor photons (DAem) during donor excitation (Dex) and only acceptor photons (Aem) during acceptor excitation (Aex)}.

\begin{table}
\begin{tabular}{l|l}
  Photon selection  & code \\
  \hline
  All-photons       & \verb|Ph_sel('all')|\\
  D-emission    & \verb|Ph_sel(Dex='Dem')|\\
  A-emission & \verb|Ph_sel(Dex='Aem')|\\
\end{tabular}
\caption{\label{tab:ph_sel_smfret}Photon selection syntax (non-ALEX)}
\end{table}

\begin{table}
\begin{tabular}{l|l}
  Photon selection  & code \\
  \hline
  All-photons & \verb|Ph_sel('all')|\\
  D-emission during D-excitation & \verb|Ph_sel(Dex='Dem')|\\
  A-emission during D-excitation & \verb|Ph_sel(Dex='Aem')|\\
  D-emission during A-excitation & \verb|Ph_sel(Aex='Dem')|\\
  A-emission during A-excitation & \verb|Ph_sel(Aex='Aem')|\\
\end{tabular}
\caption{\label{tab:ph_sel_alex}Photon selection syntax (ALEX)}
\end{table}

\subsection{Background periods}
\label{sec:bg_intro}

Even when no molecule is crossing the excitation volume, there are “background counts” due to detectors dark counts, sample scattering and auto-fluorescence. Figure~\ref{fig:bgdist} shows the typical distribution of timestamps delays (i.e. the waiting times between two subsequent timestamps) in a smFRET measurement. The “tail” of the distribution (a line in semi-log scale) corresponds to exponentially-distributed delays, indicating that those counts are generated by a \href{http://en.wikipedia.org/wiki/Poisson_process}{Poisson process}. At short timescales, the distribution departs from the exponential due to the bursts of photons from diffusing single-molecules (the signal). To estimate the background rate, (i.e. the exponential time constant) we need to select a minimum threshold above which the distribution can be considered exponential. We also need to chose a fit method, for example the Maximum Likelihood Estimation (MLE) or a curve fit of the histogram via non-linear least squares (NLSQ).

Both burst search and burst correction require background rates for all the different photon streams. Furthermore, we want to estimate the background periodically (every a few seconds) because it can varies during the measurement on time scales of tens of seconds. FRETBursts splits the data in uniform time slices called \textit{background periods} and compute the background rates for each of these slices (see section~\ref{sec:bg_calc}). The slicing in background periods is also used during burst search to compute a background-dependent threshold and to apply the burst correction (section~\ref{sec:burstsearch}).

\subsection{The \texttt{Data} class}
\label{sec:data_intro}

The \href{http://fretbursts.readthedocs.org/en/latest/data_class.html}{\texttt{Data} class} is the fundamental data container in FRETBursts. It contains the measurement data and provides several methods for data analysis (background estimation, burst search, etc...). It also stores all the analysis results (bursts data, estimated parameters).

All the arrays in Data are contained in lists whose length is equal to the number of excitation spots. This means that for single-spot measurements all the arrays are wrapped in 1-element lists. For example, the bursts data field \verb|Data.mbursts| will be a 1-element list and \verb|Data.mbursts[0]| will be the actual numpy array of burst data. \verb|Data|implements a shortcut syntax that allows accessing 
\verb|Data.mbursts[0]| as \verb|Data.mbursts_| (valid for all the fields).

As an example the following are some important burst-data fields:

\begin{itemize}
\item \verb|nd|: number of photons detected by the donor channel (during donor excitation), after correction
\item \verb|na|: number of photons detected by the acceptor channel (during donor excitation), after correction
\item \verb|naa|: number of photons detected by the acceptor channel during acceptor excitation, after correction
\end{itemize}

\subsection{Plotting "Data"}

FRETBursts provide a wide range of built-in plot functions for \verb|Data| objects. The same syntax is used both for single and multi-spot measurements. Almost all the plot commands are called through the wrapper function \verb|dplot|, for example to plot a timetrace of the photon data we type:

\begin{verbatim}
dplot(d, timetrace)
\end{verbatim}

The function \verb|dplot| is the generic plot function that creates the figure and handles details common to all the plotting functions (i.e. the title). \verb|d| is the \verb|Data| variable and \verb|timetrace| is the actual plot function that operates on a single channel. In multi-spot measurements \verb|dplot| creates one subplot for each spot and calls \verb|timetrace| for each channel.

All the built-in plot functions that can be passed to \verb|dplot| are defined in the \verb|burst_plot| module. When importing fretbursts all the plot functions are also imported. To make easy to find plot function through auto-completion, all the plot functions names start with the plot type. The plot names prefixes are: \verb|timetrace| for binned timetraces of photon data, \verb|ratetrace| for rates of photons as a function of time (non binnned), \verb|hist| for functions plotting histograms and \verb|scatter| for scatter plots.

