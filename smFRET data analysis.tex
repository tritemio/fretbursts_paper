\section{smFRET data analysis}

In the following we provide a high-level description of FRETbursts usage. Normally, especially when FRETbursts is combined with other packages, we suggest to import FRETBursts with:

\verb|import fretbursts as fb|

In this article, however, to save some typing we assume that FRETBursts is imported with:

\verb|from fretbursts import *|

In this way we can avoid to use the \verb|fb.| prefix to access FRETBursts functions. Note also that the previous import will also import numpy, pandas, matplotlib.pyplot (as \verb|np|, \verb|pd| and \verb|plt| respectively).

\subsection{Loading the data}
Currently FRETBursts supports loading data from a few file formats: SM files (a binary format saved by a common LabVIEW program used in smFRET setups), SPC a binary format used by TCSPC Becker & Hickl cards and HDF5-Ph-Data an open binary format single-molecule data based on HDF5.

We encourage adopting HDF5-Ph-Data format beacuse it is simple, scalable, well documented, space efficient and very fast. All the FRETBursts example data files are in HDF5-Ph-Data. An additional file-set can be found here[multi-spot paper data files].

Functions to load data files are defined in the \verb|loader| module. In particular, to load an HDF5-Ph-Data file use:

\verb|d = loader.hdf5(file_name)|

where \verb|file_name| is a string containing the file path. In the previous command, the variable `d` (that is an object of type Data) contains the measurement data and several methods to process it. 
If you have smFRET data in another format, you can manually create a `Data` variable. For example, for non-ALEX smFRET data, you need  to create two arrays of same length: timestamps and acceptor-mask. The timestamps is an int64 numpy array containing the recorded photon timestamps in arbitrary units (usually dictated by the acquisition hardware clock period). The acceptor-mask is a numpy boolean array that is True when the corresponding timestamps comes from the acceptor channel and False when it comes from the donor channel. Once you have these arryas you can manually create a `Data` object with:

\verb|d = Data(ph_times_m=[timestamps], A_em=[acceptor_mask], clk_p=10e-9, ALEX=False)|

In the previous example, we set the timestamp unit (`clk_p`) to 10 ns and we specify that the data is not from an ALEX measurements. Creating Data objects for ALEX and nsALEX measurements follows the same lines. We point the interested readers to the loader module for the details.

On MS Windows, it is good practice to use RAW strings for file names (for example: \verb|r'C:\Data\smFRET\example.hdf5'|, note the prepending \textit{r}) in order to avoid substitutions of special escape sequences like '\t' (that would be replaced with TAB in a normal string).

\subsection{Background estimation}
\begin{itemize}
\item Background as a function of time
\end{itemize}

\subsubsection{Choice of the threshold}
\begin{itemize}
\item heuristic
\item brute force
\end{itemize}

\subsection{Burst search}

Description of burst-search algorithms and why the m-photons sliding windows is exactly the same as fixed-time sliding window. Maybe a picture will help.

\begin{itemize}
\item Adaptive threshold as a function of background
\item Chosing different photon streams
\item AND-Gate
\end{itemize}

\subsection{Burst selection}

How to select bursts according to different criteria (size, width, E, S, topN, etc...).

How to define a new criterium.


\subsection{Population fit}

\begin{itemize}
\item Histogram fit: chose a model, constraints, methods, accuracy
\item KDE: find the maximum
\end{itemize}

\subsubsection{Correction coefficients}

\begin{itemize}
\item Fit D-only and A-only population.
\item Fit gamma factor.
\end{itemize}


\subsubsection{Accurate FRET}

Apply corrections to the bursts vs apply corrections after the fit.

