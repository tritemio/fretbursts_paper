\section{Conclusions}
\label{sec:conclusions}

FRETBursts is an open source and openly developed (see SI~\ref{sec:dev}) implementation
of established smFRET burst analysis methods
made available to the single-molecule community.
It implements several novel concepts which improve the analysis results, such as
time-dependent background estimation, background-dependent burst search threshold,
burst weighting and $\gamma$-corrected burst size selection.
More importantly, FRETBursts provides a library of thoroughly-tested functions
for timestamps and burst manipulation, making it an ideal platform for
developing and comparing new analytical techniques.

We envision FRETBursts both as a state-of-the-art burst analysis
software as well as a platform for development and assessment of novel algorithms.
To underpin this envisioned role, FRETBursts is developed following modern
software engineering practices, such as DRY principle
(\href{http://en.wikipedia.org/wiki/Don\%27t_repeat_yourself}{link})
to reduce duplication and KISS principle
(\href{http://en.wikipedia.org/wiki/KISS_principle}{link})
to reduce over-engineering. Furthermore, to minimize the number software errors~\cite{Merali_2010,Soergel_2015},
we employ defensive programming~\cite{Prli__2012} which includes code readability,
unit and regression testing and continuous integration~\cite{Eglen_2016}.
Finally, being open source, any scientist can inspect the source code,
fix errors, adapt it to her own needs.

We believe that, in the single-molecule community,
standard open source software implementations, such as FRETBursts, can enhance
reliability and reproducibility of analysis and promote a faster adoption of novel methods,
while reducing the duplication of efforts among different groups.

\section*{Acknowledgments}
We thank Dr. Eyal Nir and Dr. Toma Tomov for support in the implementation of the 2CDE method.
This work was supported by National Institutes of Health (NIH)
grant R01-GM95904 and R01-GM069709. Dr. Weiss discloses equity in
Nesher Technologies and intellectual property used in the research
reported here. The work at UCLA was conducted in Dr. Weiss’s Laboratory. 
