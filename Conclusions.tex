\section{Conclusions}
\label{sec:conclusions}

We have described the usage of the numerous algorithms in FRETBursts
for burst analysis. We conclude enumerating what we consider to be the strengths
of this software.

\begin{enumerate}
\item Open source and openly developed. The source code can be checked, modified and
adapted for different purposes. All the software dependencies are open source as well.
\item Provides several state-of-the-art and novel algorithms for each step of the 
smFRET burst analysis pipeline. Examples of novel concepts implemented in FRETBursts are background estimation as a function of time,
background-dependent threshold for burst search, burst filtering based on 
gamma-corrected burst sizes and burst weighting.
\item Modern software engineering design: we follows the \href{http://en.wikipedia.org/wiki/Don\%27t_repeat_yourself}{DRY principle} to reduce duplication and the \href{http://en.wikipedia.org/wiki/KISS_principle}{KISS principle} to avoid over-engineering.
\item Defensive programming~\cite{Prli__2012}: we aim for the maximum code readability,
extensive unit and regression testing and continuous integration.
\end{enumerate}

Given these features FRETBursts is suitable both as a toolkit to develop novel algorithms
in smFRET burst analysis or as a software for processing smFRET data files with
state-of-the art algorithms.
