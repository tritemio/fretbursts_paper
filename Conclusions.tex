\section{Conclusions}
\label{sec:conclusions}

FRETBursts is an open source implementation of state-of-the-art smFRET burst analysis methods
accessible to the whole single-molecule community.
It implements several novel concepts which can lead
to significantly more accurate results. Example of novel concepts include
time-dependent background estimation, background dependent burst search threshold,
burst weighting and burst selection based on γ-corrected burst sizes.

More importantly, FRETBursts provides a library of well-tested routines
for timestamps and burst manipulation, making it an ideal environment to
develop and compare novel analytical techniques with minimal effort.

We summarize here what we consider to be the strengths
of the FRETBursts software.

\begin{enumerate}
\item Open source and openly developed. The source code can be inspected, modified and
adapted for different purposes. All the software dependencies are open source as well.
\item State-of-the-art and novel algorithms for each step of the
smFRET burst analysis pipeline.
\item Modern software engineering design (e.g. DRY principle
(\href{http://en.wikipedia.org/wiki/Don\%27t_repeat_yourself}{link})
to reduce duplication and KISS principle
(\href{http://en.wikipedia.org/wiki/KISS_principle}{link})
to reduce over-engineering).
\item Defensive programming~\cite{Prli__2012}: code readability,
unit and regression testing and continuous integration.
\end{enumerate}

We envision FRETBursts both as a standard burst analysis
software as well as a platform for development and assessment of novel algorithms.
We believe that a standard software implementation can improve
reproducibility and promote a faster adoption of novel methods 
while reducing the duplication of efforts among different groups
in the single-molecule community.
Finally, FRETBursts open source nature, guarantees that the source code
can always be inspected, fixed and improved
by any member of the community.


\section*{Acknowledgments}
We thank Prof. Eyal Nir for discussions and clarifications regarding the 
implementation of the 2CDE method.
This work was supported in part by National Institutes of Health (NIH)
grant R01-GM95904 and by U.S. Department Energy (DOE) grant DEFC02-02ER63421-00.
