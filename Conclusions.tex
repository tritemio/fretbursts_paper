\section{Conclusions}
\label{sec:conclusions}

FRETBursts is an open source implementation of established smFRET burst analysis methods
accessible to the single-molecule community.
It implements several novel concepts which improve the analysis results, such as
time-dependent background estimation, background-dependent burst search threshold,
burst weighting and γ-corrected burst size selection.

More importantly, FRETBursts provides a library of thoroughly-tested functions
for timestamps and burst manipulation, making it an ideal platform to
develop and compare new analytical techniques with minimal effort.

We summarize here what we consider to be the strengths
of the FRETBursts software.

\begin{enumerate}
\item Open source and openly developed (see SI~\ref{sec:dev}).
All the software dependencies are open source as well.
\item State-of-the-art and novel algorithms for each step of the
smFRET burst analysis pipeline.
\item Modern software engineering design (e.g. DRY principle
(\href{http://en.wikipedia.org/wiki/Don\%27t_repeat_yourself}{link})
to reduce duplication and KISS principle
(\href{http://en.wikipedia.org/wiki/KISS_principle}{link})
to reduce over-engineering).
\item Defensive programming~\cite{Prli__2012}: code readability,
unit and regression testing and continuous integration.
\end{enumerate}

We envision FRETBursts both as a state-of-the-art burst analysis
software as well as a platform for development and assessment of novel algorithms.
We believe that, in the single-molecule community,
a standard software implementation can enhance
reliability and reproducibility of analysis and promote a faster adoption of novel methods, 
while reducing the duplication of efforts among different groups.
Finally, FRETBursts open source nature, guarantees that the source code
can always be inspected, fixed and improved by any member of the community.

\section*{Acknowledgments}
We thank Dr. Eyal Nir and Dr. Toma Tomov for discussions and clarifications regarding the 
implementation of the 2CDE method.
This work was supported by National Institutes of Health (NIH)
grant R01-GM95904 and in part by U.S. Department Energy (DOE) grant DEFC02-02ER63421-00.
