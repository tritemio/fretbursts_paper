\subsection{Plotting \texttt{Data}}
\label{sec:plotting}

FRETBursts uses
matplotlib~\cite{matplotlib}
and seaborn~\cite{seaborn}
to provide a wide range of built-in plot functions
(\href{http://fretbursts.readthedocs.org/en/latest/plots.html}{link})
for \verb|Data| objects.
The plot syntax is the same for both single and multi-spot measurements.
The majority of plot commands are called through the wrapper function
\verb|dplot|, for example to plot a timetrace of the photon data, type:

\begin{lstlisting}
dplot(d, timetrace)
\end{lstlisting}

The function \verb|dplot| is the generic plot function, which creates figure
and handles details common to all the plotting functions (for instance, the title).
\verb|d| is the \verb|Data| variable and \verb|timetrace| is the actual plot
function, which operates on a single channel. In multispot measurements,
\verb|dplot| creates one subplot for each spot and calls \verb|timetrace| for
each channel.

All built-in plot functions which can be passed to
\verb|dplot| are defined in the
\verb|burst_plot| module
(\href{http://fretbursts.readthedocs.org/en/latest/plots.html}{link}).

\paragraph{Python details}
When FRETBursts is imported, all plot functions are also imported.
To facilitate finding the plot functions through auto-completion,
their names start with a standard prefix indicating the
plot type. The prefixes are: \verb|timetrace| for binned timetraces
of photon data, \verb|ratetrace| for rates of photons as a function of time (non
binnned), \verb|hist| for functions plotting histograms and \verb|scatter| for
scatter plots.
Additional plots can be easily created directly with matplotlib.

By default, in order to speed-up batch processing, FRETBursts notebooks display plots
as static images using the \textit{inline} matplotlib backend.
User can switch to interactive figures inside the browser by activating
the interactive backend with the command \verb|%matplotlib notebook|.
Another option is displaying figures in a new standalone window
using a desktop graphical library such as QT4.
In this case, the command to be used is \verb|%matplotlib qt|.

A few plot functions, such as \verb|timetrace| and \verb|hist2d_alex|, have interactive features
which require the QT4 backend. As an example, after switching to the QT4 backend
the following command:

\begin{lstlisting}
dplot(d, timetrace, scroll=True, bursts=True)
\end{lstlisting}

\noindent
will open a new window with a timetrace plot with overlay of bursts, and an horizontal scroll-bar for quick
"scrolling" throughout time. The user can click on a burst to have the corresponding burst info
be printed in the notebook.
Similarly, calling the \verb|hist2d_alex| function with the QT4 backend allows
selecting an area on the E-S histogram using the mouse.

\begin{lstlisting}
dplot(ds, hist2d_alex, gui_sel=True)
\end{lstlisting}

The values which identify the region are printed in the notebook and can be passed
to the function \verb|select_bursts.ES| to select bursts inside that region
(see section~\ref{sec:burstsel}).
