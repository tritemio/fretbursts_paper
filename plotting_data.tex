\subsection{Plotting \texttt{Data}}
\label{sec:plotting}

FRETBursts uses
matplotlib~\cite{matplotlib}
and seaborn~\cite{seaborn}
to provide a wide range of built-in plot functions
(\href{http://fretbursts.readthedocs.org/en/latest/plots.html}{link})
for \verb|Data| objects.
The plot syntax is the same for both single and multi-spot measurements.
The majority of plot commands are called through the wrapper function
\verb|dplot|, for example to plot a timetrace of the photon data, type:

\begin{lstlisting}
dplot(d, timetrace)
\end{lstlisting}

The function \verb|dplot| is the generic plot function, which creates figure
and handles details common to all the plotting functions (for instance, the title).
\verb|d| is the \verb|Data| variable and \verb|timetrace| is the actual plot
function, which operates on a single channel. In multispot measurements,
\verb|dplot| creates one subplot for each spot and calls \verb|timetrace| for
each channel.

All built-in plot functions which can be passed to
\verb|dplot| are defined in the
\verb|burst_plot| module
(\href{http://fretbursts.readthedocs.org/en/latest/plots.html}{link}).

\paragraph{Python details}
When FRETBursts is imported, all plot functions are also imported.
To facilitate finding the plot functions through auto-completion,
their names start with a standard prefix indicating the
plot type. The prefixes are: \verb|timetrace| for binned timetraces
of photon data, \verb|ratetrace| for rates of photons as a function of time (non
binnned), \verb|hist| for functions plotting histograms and \verb|scatter| for
scatter plots.
Additional plots can be easily created directly with matplotlib.

By default, in order to speed-up batch processing, FRETBursts notebooks display plots
as static images using the \textit{inline} matplotlib backend.
User can switch to interactive figures inside the browser by activating
the interactive backend with the command \verb|%matplotlib notebook|.
Another option is displaying figures in a new standalone window
using a desktop graphical library such as QT4.
In this case, the command to be used is \verb|%matplotlib qt|.

A few plot functions, such as \verb|timetrace| and \verb|hist2d_alex|, have interactive features
which require the QT4 backend. As an example, after switching to the QT4 backend
the following command:

\begin{lstlisting}
dplot(d, timetrace, scroll=True, bursts=True)
\end{lstlisting}

\noindent
will open a new window with a timetrace plot with overlay of bursts, and an horizontal scroll-bar for quick
"scrolling" throughout time. The user can click on a burst to have the corresponding burst info
be printed in the notebook.
Similarly, calling the \verb|hist2d_alex| function with the QT4 backend allows
selecting an area on the E-S histogram using the mouse.

\begin{lstlisting}
dplot(ds, hist2d_alex, gui_sel=True)
\end{lstlisting}

The values which identify the region are printed in the notebook and can be passed
to the function \verb|select_bursts.ES| to select bursts inside that region
(see section \textit{Burst Selection} in the main text).

\subsection*{Plotting ALEX histograms}
E-S histograms are traditionally computed using a bin size of 0.02-0.04, 
and cover a range slightly larger than the [0, 1] interval in which ratios of 
quantities not corrected for background would normally fall. 
FRETBursts allows plotting the square-bin 2-D E-S histogram using the plot
function \verb|hist2d_alex| as shown in the previous example.
The histogram can be ``smothed'' via bicubic interpolation between bin centers
(default) or plotted with raw square bins (pass the argument 
\verb|interpolation='none'|). Additionally, a scatter plot of E-S
points is by default overlayed and it can be useful for observing
the burst distribution in sparse regions. However, the different layers
make this plot hard to read. 

A more elegant approach for effectively representing E-S histograms with minimal
clutter and high information content is using an hexbin plot 
(as used by the FRETBursts function \verb|alex_jointplot|). The hexbin plot 
is a 2-D histograms using hexagonal bins that reduces
gridding artifacts of square bins. In addition, in sparse regions, the hexbin 
plot naturally resembles a scatter plot (with hexagonal markers).
The use of hexagonal bins for 2D distributions 
has been pioneered by Dan Carr in S-PLUS and then popularized by Nicholas Lewin-Koh
which wrote the R language port. Later, hexbin has been implemented in the 
matplotlib python library (which is what FRETBursts uses). The advantages of 
hexagonal bins have been extensively studied and can be summarized with 
Nicholas Lewin-Koh words (\href{https://cran.r-project.org/web/packages/hexbin/vignettes/hexagon_binning.pdf}{link}):

\begin{quote}
Why hexagons? There are many reasons for using hexagons, at least over squares. 
Hexagons have symmetry of nearest neighbors which is lacking in square bins. 
Hexagons are the maximum number of sides a polygon can have for a regular 
tesselation of the plane, so in terms of packing a hexagon is 13\% more 
efficient for covering the plane than squares. This property translates into 
better sampling efficiency at least for elliptical shapes. Lastly hexagons are 
visually less biased for displaying densities than other regular tesselations. 
\end{quote}

The function \verb|alex_jointplot| plots a 3-panels plots with a central
hexbin plot of E-S values and marginal E and S histograms represented in
top and right panel (see figure~4 and 5 in the main text). Note that
unlike other plot functions in FRETBursts, \verb|alex_jointplot| 
is called directly and not through the \verb|dplot| wrapper.