\section{Development and Contributions}
\label{sec:dev}

Errors are an inevitable reality in any reasonably complex software. It is 
therefore critical to implement countermeasures to
minimize the probability of introducing bugs and their potential impact~\cite{Prli__2012, Wilson_2014}.
We strive to follow modern best-practices in software development which are summarized 
below.

In FRETBursts, we highly value source code readability, a property which can 
reduce the number of bugs by facilitating understanding and verifying the code.
For this purpose, FRETBursts code-base is well commented (more that 35\%
of source code), 
follows the PEP8 python code style rules (\href{https://www.python.org/dev/peps/pep-0008/}{link}),
and has docstrings in napoleon format (\href{http://sphinxcontrib-napoleon.readthedocs.org/}{link}).

Reference documentation is built with Sphinx (\href{http://sphinx-doc.org/}{sphinx-doc.org})
and all the API documents are automatically generated from docstrings.
On each commit, documentation is automatically built and deployed on
\href{https://readthedocs.org/}{ReadTheDocs.org}.

Unit tests cover most of the core algorithms, ensuring consistency and
minimizing the probability of introducing bugs. 
The TravisCI (\href{http://travis-ci.org}{link}) continuous integration service, 
executes the full test suite on each commit, timely reporting errors.
As a rule, whenever a bug is discovered, the  fix also includes a new test
to ensure that the same bug cannot happen in the future.
In addition to the unit tests, we include a regression-test notebook
(\href{https://github.com/tritemio/FRETBursts/blob/master/notebooks/dev/tests/FRETBursts%20-%20Regression%20tests.ipynb}{link})
to easily compares numerical results between two versions of FRETBursts. 
Additionally, the tutorials themselves are executed before each release as
an additional test layer to ensure that no errors or regressions are introduced.

FRETBursts is openly developed using the GitHub platform.
Contributors of any level of expertize are welcome in the projects
and publically acknowledged.
Contributions can be as simple as pointing out deficiencies in the 
documentation but can also be bug reports or corrections to 
the documentation or code. Users willing to implement
new features are encouraged to open an Issue on GitHub and to submit
a Pull Request. The open source nature of FRETBursts guarantees that
contributions will remain available to the entire single-molecule 
community.

